\DeclareRBKTerm{release_gate}{Gate GO/NO-GO}{Projet/Management/Qualité}{Point de décision formalisé (GO/NO-GO) après vérification build, scripts et absence de hardcodes ; bloque la promotion en cas d’échec.}
\DeclareRBKTerm{linkedin}{LinkedIn}{Organisation}{Réseau professionnel utilisé pour la visibilité, l’acquisition de leads et le recrutement de partenaires ou candidats.}
\DeclareRBKTerm{sec_lead}{SecLead}{Organisation}{Responsable sécurité pilotant analyse de risques, contrôles et plans d’actions lors des projets ou incidents.}
\DeclareRBKTerm{tech_lead}{TechLead}{Organisation}{Référent technique coordonnant architecture, qualité et mentoring d’équipe pour livrer un produit robuste.}
\DeclareRBKTerm{senior_by_design}{Senior-by-Design}{Organisation}{Principe de conception du programme visant à inculquer dès la formation les réflexes et standards d’un profil senior.}
% ------------------------------------------------------------
% Terminologie RBK — Entrées centralisées (SSOT des définitions)
% ------------------------------------------------------------

% Web3 / Blockchain
\DeclareRBKTerm{dao}{DAO}{Web3/Blockchain}{Organisation autonome décentralisée gouvernée par des règles codées on-chain où les détenteurs de jetons votent les décisions et les budgets via des propositions transparentes.}
\DeclareRBKTerm{nft}{NFT}{Web3/Blockchain}{Jeton non fongible représentant un actif unique dont la propriété et l’historique sont inscrits sur la blockchain, avec une métadonnée traçable.}
\DeclareRBKTerm{defi}{DeFi}{Web3/Blockchain}{Finance décentralisée construite sur des smart contracts, offrant prêts, échanges et rendements sans intermédiaire bancaire traditionnel.}
\DeclareRBKTerm{dex}{DEX}{Web3/Blockchain}{Exchange décentralisé permettant l’échange pair-à-pair de tokens via des pools de liquidité ou des carnets d’ordres on-chain.}
\DeclareRBKTerm{amm}{AMM}{Web3/Blockchain}{Automated Market Maker utilisant des formules algorithmiques et des pools de liquidité pour fixer les prix sans carnet d’ordres central.}
\DeclareRBKTerm{dapp}{dApp}{Web3/Blockchain}{Application décentralisée dont la logique métier principale repose sur des smart contracts et une interface utilisateur web ou mobile.}
\DeclareRBKTerm{layer1}{Layer 1}{Web3/Blockchain}{Blockchain de base assurant consensus, exécution et données, sur laquelle des protocoles ou applications peuvent être construits.}
\DeclareRBKTerm{layer2}{Layer 2}{Web3/Blockchain}{Solution d’extension de capacité bâtie au-dessus d’une chaîne de base pour améliorer débit ou coûts tout en héritant de sa sécurité.}
\DeclareRBKTerm{rollup}{Rollup}{Web3/Blockchain}{Regroupement de transactions hors chaîne avec publication régulière d’états ou de preuves sur la chaîne principale pour sécuriser l’ensemble.}
\DeclareRBKTerm{bridge}{Bridge}{Web3/Blockchain}{Mécanisme de transfert ou de verrouillage d’actifs entre blockchains distinctes, basé sur des preuves ou des validateurs fédérés.}
\DeclareRBKTerm{oracle}{Oracle}{Web3/Blockchain}{Service apportant des données hors chaîne fiables aux smart contracts afin de déclencher des actions selon des événements du monde réel.}
\DeclareRBKTerm{zk_proof}{Preuve ZK}{Web3/Blockchain}{Preuve à connaissance nulle démontrant la validité d’une information sans révéler les données sous-jacentes, utilisée pour la confidentialité ou l’évolutivité.}
\DeclareRBKTerm{optimistic_rollup}{Optimistic rollup}{Web3/Blockchain}{Rollup supposant les blocs valides par défaut et offrant une période de contestation où des preuves de fraude peuvent être soumises.}
\DeclareRBKTerm{zk_rollup}{ZK rollup}{Web3/Blockchain}{Rollup publiant des preuves cryptographiques succinctes pour garantir l’exactitude des transitions d’état tout en réduisant les données on-chain.}
\DeclareRBKTerm{rpc}{RPC}{Web3/Blockchain}{Interface Remote Procedure Call exposant des points de terminaison pour lire l’état de la blockchain ou soumettre des transactions.}
\DeclareRBKTerm{validator}{Validateur}{Web3/Blockchain}{Noeud participant au consensus en produisant ou attestant des blocs selon les règles du protocole, souvent contre une mise sous caution.}
\DeclareRBKTerm{staking}{Staking}{Web3/Blockchain}{Blocage de tokens pour sécuriser un réseau Proof-of-Stake et recevoir des récompenses en échange de la participation au consensus.}
\DeclareRBKTerm{slashing}{Slashing}{Web3/Blockchain}{Pénalité appliquée aux validateurs fautifs ou inactifs entraînant la réduction de leur mise et une exclusion temporaire ou définitive.}
\DeclareRBKTerm{gas_fee}{Gas fee}{Web3/Blockchain}{Frais payé pour exécuter une transaction ou un smart contract, reflétant la consommation de ressources réseau.}
\DeclareRBKTerm{tps}{TPS}{Web3/Blockchain}{Transactions par seconde mesurant le débit effectif d’un réseau blockchain dans des conditions réelles.}
\DeclareRBKTerm{mempool}{Mempool}{Web3/Blockchain}{Zone tampon où les transactions en attente sont stockées avant d’être incluses dans un bloc par un validateur ou un mineur.}
\DeclareRBKTerm{block_explorer}{Block Explorer}{Web3/Blockchain}{Outil web permettant de consulter blocs, transactions, comptes et événements on-chain avec des filtres et des recherches.}
\DeclareRBKTerm{multisig}{Multisig}{Web3/Blockchain}{Portefeuille ou smart contract exigeant plusieurs signatures indépendantes pour autoriser une transaction, réduisant le risque de compromission.}
\DeclareRBKTerm{wallet}{Wallet}{Web3/Blockchain}{Portefeuille numérique gérant clés privées, adresses et interactions avec des dApps pour signer des transactions.}
\DeclareRBKTerm{seed_phrase}{Seed phrase}{Web3/Blockchain}{Suite de mots permettant de dériver et restaurer toutes les clés d’un portefeuille hiérarchique déterministe.}
\DeclareRBKTerm{hardware_wallet}{Hardware wallet}{Web3/Blockchain}{Portefeuille matériel isolant les clés privées dans un dispositif physique pour signer hors ligne et limiter l’exposition aux malwares.}
\DeclareRBKTerm{siws}{SIWS}{Web3/Blockchain}{Sign-In With Solana permettant une authentification décentralisée par signature cryptographique plutôt que par mot de passe.}
\DeclareRBKTerm{smart_contract}{Smart contract}{Web3/Blockchain}{Programme déployé sur blockchain exécutant automatiquement des règles vérifiables, immuables et auditées par le réseau.}
\DeclareRBKTerm{pda}{PDA}{Web3/Blockchain}{Program Derived Address de Solana, adresse déterministe contrôlée par un programme plutôt que par une clé privée humaine.}
\DeclareRBKTerm{program_solana}{Programme Solana}{Web3/Blockchain}{Binaire on-chain compilé pour la VM Solana exécutant des instructions déterministes et gérant l’état via des comptes.}
\DeclareRBKTerm{tokenomics}{Tokenomics}{Web3/Blockchain}{Conception économique d’un token couvrant émission, distribution, utilité, incitations et mécanismes de capture de valeur.}
\DeclareRBKTerm{governance_token}{Governance token}{Web3/Blockchain}{Jeton offrant des droits de vote ou des droits de proposition pour orienter l’évolution d’un protocole ou d’une DAO.}
\DeclareRBKTerm{utility_token}{Utility token}{Web3/Blockchain}{Jeton fournissant un droit d’usage ou d’accès à un service sans nécessairement conférer des droits de gouvernance.}
\DeclareRBKTerm{stablecoin}{Stablecoin}{Web3/Blockchain}{Actif numérique cherchant à maintenir une parité stable via collatéral, algorithme ou réserves hors chaîne auditées.}
\DeclareRBKTerm{liquidity_pool}{Pool de liquidité}{Web3/Blockchain}{Réserve de tokens déposée par des fournisseurs de liquidité pour faciliter les échanges et percevoir des frais ou récompenses.}
\DeclareRBKTerm{yield_farming}{Yield farming}{Web3/Blockchain}{Stratégie consistant à déplacer des actifs entre protocoles pour optimiser les récompenses de liquidité ou d’incitation.}
\DeclareRBKTerm{airdrop}{Airdrop}{Web3/Blockchain}{Distribution gratuite ou ciblée de tokens à une communauté pour récompenser l’usage ou stimuler l’adoption.}
\DeclareRBKTerm{metaplex}{Metaplex}{Web3/Blockchain}{Suite d’outils et de standards Solana facilitant la création, la gestion et la vente de NFT et collections.}
\DeclareRBKTerm{umi}{UMI}{Web3/Blockchain}{Bibliothèque Solana pour construire et signer des transactions de manière sûre et modulaire côté client ou serveur.}
\DeclareRBKTerm{sbt}{SBT}{Web3/Blockchain}{Soulbound Token non transférable, attaché à une identité ou un compte pour prouver un statut, une attestation ou une réussite.}
\DeclareRBKTerm{onchain}{On-chain}{Web3/Blockchain}{Donnée ou logique exécutée et stockée directement sur la blockchain, vérifiée par le consensus du réseau.}
\DeclareRBKTerm{offchain}{Off-chain}{Web3/Blockchain}{Donnée ou traitement réalisé hors blockchain, pouvant être ancré ou attesté ensuite on-chain via oracle ou preuve.}
\DeclareRBKTerm{whitelist}{Whitelist}{Web3/Blockchain}{Liste d’adresses ou d’identités autorisées à interagir avec un contrat, souvent utilisée pour les mint ou accès anticipés.}
\DeclareRBKTerm{erc}{ERC}{Web3/Blockchain}{Ethereum Request for Comments : famille de standards décrivant des interfaces de jetons et contrats (ex. ERC-20, ERC-721) dans l’écosystème Ethereum.}
\DeclareRBKTerm{rust}{Rust}{Web3/Blockchain}{Langage de programmation performant et sûr en mémoire, largement utilisé pour développer des programmes on-chain sur Solana.}
\DeclareRBKTerm{anchor}{Anchor}{Web3/Blockchain}{Framework Solana en Rust facilitant IDL, validation et écriture de programmes on-chain avec conventions et outillage.}
\DeclareRBKTerm{id}{ID}{Ingénierie}{Identifiant unique permettant de référencer une ressource, un utilisateur ou un objet de manière non ambiguë.}
\DeclareRBKTerm{idl}{IDL}{Web3/Blockchain}{Interface Definition Language décrivant formellement fonctions et types d’un programme pour générer des clients et intégrer un contrat.}
\DeclareRBKTerm{mttr}{MTTR}{SRE/Ops}{Mean Time To Recovery : temps moyen nécessaire pour rétablir un service après incident.}
\DeclareRBKTerm{svm}{SVM}{Web3/Blockchain}{Sealevel Virtual Machine : runtime d’exécution Solana gérant le parallélisme et les contraintes de performances on-chain.}
\DeclareRBKTerm{uups}{UUPS}{Web3/Blockchain}{Universal Upgradeable Proxy Standard : pattern proxy EVM permettant de mettre à jour la logique d’un contrat via une implémentation remplaçable.}
\DeclareRBKTerm{cu}{CU}{Web3/Blockchain}{Compute Units : unité mesurant les ressources de calcul consommées lors de l’exécution on-chain sur Solana.}
\DeclareRBKTerm{evm}{EVM}{Web3/Blockchain}{Ethereum Virtual Machine exécutant des smart contracts bytecode compatibles sur Ethereum et chaînes EVM, assurant portabilité des dApps et outils.}
\DeclareRBKTerm{depin}{DePIN}{Web3/Blockchain}{Decentralized Physical Infrastructure Network incitant via tokens la construction et l’exploitation collective d’infrastructures physiques (connectivité, capteurs, énergie).}
\DeclareRBKTerm{cross_chain}{cross-chain}{Web3/Blockchain}{Capacité d’un système à interagir entre plusieurs blockchains (transferts, messages, états) via des ponts ou protocoles d’interopérabilité.}
\DeclareRBKTerm{erc20}{ERC-20}{Web3/Blockchain}{Standard Ethereum définissant l’interface minimale d’un jeton fongible (fonctions de transfert, approbation, total supply).}

% Sécurité / Infra / Ops
\DeclareRBKTerm{kyc}{KYC}{Sécurité/Infra/Ops}{Processus de connaissance client vérifiant l’identité avant d’accéder à un service financier ou réglementé.}
\DeclareRBKTerm{aml}{AML}{Sécurité/Infra/Ops}{Lutte contre le blanchiment d’argent couvrant surveillance des flux, filtrage des sanctions et déclarations aux autorités.}
\DeclareRBKTerm{zero_trust}{Zero Trust}{Sécurité/Infra/Ops}{Modèle de sécurité où aucune entité n’est implicitement fiable et où chaque accès est continuellement vérifié et limité au strict nécessaire.}
\DeclareRBKTerm{iam}{IAM}{Sécurité/Infra/Ops}{Gestion des identités et des accès assurant authentification, autorisation et gouvernance des droits dans le SI.}
\DeclareRBKTerm{mfa}{MFA}{Sécurité/Infra/Ops}{Authentification multifacteur combinant au moins deux preuves distinctes pour réduire le risque de compromission.}
\DeclareRBKTerm{hsm}{HSM}{Sécurité/Infra/Ops}{Hardware Security Module protégeant clés cryptographiques et opérations de signature dans un environnement matériel certifié.}
\DeclareRBKTerm{kms}{KMS}{Sécurité/Infra/Ops}{Service de gestion de clés centralisé offrant génération, rotation, chiffrement et journalisation des usages.}
\DeclareRBKTerm{secret_management}{Gestion des secrets}{Sécurité/Infra/Ops}{Pratiques et outils pour stocker, distribuer et auditer les secrets applicatifs sans exposition en clair.}
\DeclareRBKTerm{drp}{Plan de reprise}{Sécurité/Infra/Ops}{Plan de reprise après sinistre définissant procédures, rôles et priorités pour restaurer les services après incident majeur.}
\DeclareRBKTerm{rto}{RTO}{Sécurité/Infra/Ops}{Recovery Time Objective définissant le délai maximal acceptable de restauration d’un service après interruption.}
\DeclareRBKTerm{rpo}{RPO}{Sécurité/Infra/Ops}{Recovery Point Objective définissant la perte de données maximale acceptable entre deux sauvegardes ou réplications.}
\DeclareRBKTerm{business_continuity}{PCA}{Sécurité/Infra/Ops}{Plan de continuité d’activité décrivant les dispositifs préventifs et les scénarios de maintien de service en cas de crise.}
\DeclareRBKTerm{incident_response}{Incident Response}{Sécurité/Infra/Ops}{Processus structuré pour détecter, contenir, éradiquer et documenter un incident de sécurité.}
\DeclareRBKTerm{playbook}{Playbook}{Sécurité/Infra/Ops}{Procédure détaillée et réutilisable pour traiter un type d’incident ou d’alerte de manière standardisée.}
\DeclareRBKTerm{runbook}{Runbook}{Sécurité/Infra/Ops}{Guide opérationnel décrivant les étapes techniques pour exécuter une tâche récurrente ou un diagnostic.}
\DeclareRBKTerm{rbac}{RBAC}{Sécurité/Infra/Ops}{Role-Based Access Control limitant les privilèges par rôles prédéfinis pour réduire la surface d’attaque.}
\DeclareRBKTerm{oauth}{OAuth}{Sécurité/Infra/Ops}{Cadre d’autorisation déléguée permettant à une application d’accéder à des ressources au nom d’un utilisateur sans exposer le mot de passe.}
\DeclareRBKTerm{jwt}{JWT}{Sécurité/Infra/Ops}{JSON Web Token signé (et éventuellement chiffré) transportant des claims pour l’authentification ou l’autorisation stateless.}
\DeclareRBKTerm{rate_limit}{Rate limiting}{Sécurité/Infra/Ops}{Mécanisme plafonnant le nombre de requêtes sur une période pour protéger API et services des abus.}
\DeclareRBKTerm{sre}{SRE}{Sécurité/Infra/Ops}{Site Reliability Engineering : pratiques mêlant dev et ops pour fiabiliser la production via automatisation et observabilité.}
\DeclareRBKTerm{soc2}{SOC 2}{Sécurité/Infra/Ops}{Référentiel d’audit évaluant la sécurité, la disponibilité, l’intégrité, la confidentialité et la protection des données d’un fournisseur de services.}
\DeclareRBKTerm{iso27001}{ISO 27001}{Sécurité/Infra/Ops}{Norme internationale définissant les exigences d’un système de management de la sécurité de l’information.}
\DeclareRBKTerm{siem}{SIEM}{Sécurité/Infra/Ops}{Security Information and Event Management agrégeant logs et alertes pour corréler, détecter et investiguer les menaces.}
\DeclareRBKTerm{soar}{SOAR}{Sécurité/Infra/Ops}{Security Orchestration, Automation and Response automatisant des actions de défense à partir de signaux de sécurité.}
\DeclareRBKTerm{cspm}{CSPM}{Sécurité/Infra/Ops}{Cloud Security Posture Management identifiant et corrigeant les mauvaises configurations cloud en continu.}
\DeclareRBKTerm{cwpp}{CWPP}{Sécurité/Infra/Ops}{Cloud Workload Protection Platform sécurisant charges de travail conteneurs, VM ou functions par contrôle d’intégrité et détection comportementale.}
\DeclareRBKTerm{waf}{WAF}{Sécurité/Infra/Ops}{Pare-feu applicatif filtrant les requêtes HTTP pour bloquer injections, robots ou attaques courantes côté application.}
\DeclareRBKTerm{ddos_mitigation}{Mitigation DDoS}{Sécurité/Infra/Ops}{Mesures réseau et applicatives visant à absorber ou détourner le trafic malveillant distribué saturant un service.}
\DeclareRBKTerm{stride}{STRIDE}{Sécurité/Infra/Ops}{Modèle de menaces (Spoofing, Tampering, Repudiation, Information Disclosure, Denial of Service, Elevation of Privilege) pour structurer l’analyse des risques.}
\DeclareRBKTerm{dos}{DoS}{Sécurité/Infra/Ops}{Denial of Service : attaque visant à rendre un service indisponible en saturant ses ressources ou ses dépendances.}
\DeclareRBKTerm{ops}{Ops}{Ops/SRE}{Abréviation d’Operations : pratiques et activités d’exploitation (monitoring, incidents, runbooks, disponibilité).}
\DeclareRBKTerm{observability}{Observabilité}{Sécurité/Infra/Ops}{Capacité à comprendre l’état d’un système via traces, métriques, logs et corrélations pour diagnostiquer rapidement.}
\DeclareRBKTerm{alerting}{Alerting}{Sécurité/Infra/Ops}{Mécanisme de notifications déclenchées sur des seuils ou anomalies afin d’initier une réponse rapide.}
\DeclareRBKTerm{monitoring}{Monitoring}{Sécurité/Infra/Ops}{Surveillance continue des performances et de la santé des systèmes pour détecter dérives et dégradations.}
\DeclareRBKTerm{bug_bounty}{Bug bounty}{Sécurité/Infra/Ops}{Programme incitant des chercheurs externes à signaler des vulnérabilités contre récompense encadrée.}

% Produit / API / Architecture
\DeclareRBKTerm{api}{API}{Produit/API/Architecture}{Interface de programmation exposant des services ou données pour être consommés par d’autres applications.}
\DeclareRBKTerm{openapi}{OpenAPI}{Produit/API/Architecture}{Spécification standardisée décrivant de façon contractuelle les endpoints, schémas et flux d’une API HTTP.}
\DeclareRBKTerm{rest}{REST}{Produit/API/Architecture}{Style d’architecture HTTP basé sur des ressources adressables, des verbes standards et l’absence d’état serveur côté session.}
\DeclareRBKTerm{graphql}{GraphQL}{Produit/API/Architecture}{Langage de requête permettant aux clients de demander précisément les champs souhaités sur un schéma typé.}
\DeclareRBKTerm{sdk}{SDK}{Produit/API/Architecture}{Kit de développement regroupant bibliothèques, exemples et outils pour intégrer rapidement une plateforme.}
\DeclareRBKTerm{webhook}{Webhook}{Produit/API/Architecture}{Notification HTTP sortante envoyée par un service vers une URL cliente lorsqu’un événement survient.}
\DeclareRBKTerm{event_stream}{Event stream}{Produit/API/Architecture}{Flux d’événements continus permettant de propager des changements en quasi temps réel.}
\DeclareRBKTerm{microservices}{Microservices}{Produit/API/Architecture}{Architecture découpant le système en services autonomes déployables indépendamment avec interfaces explicites.}
\DeclareRBKTerm{monolith}{Monolithe}{Produit/API/Architecture}{Application unique regroupant logique métier, données et interfaces dans un déploiement indissociable.}
\DeclareRBKTerm{hexagonal}{Architecture hexagonale}{Produit/API/Architecture}{Architecture séparant domaine, ports et adaptateurs pour isoler le coeur métier des entrées/sorties.}
\DeclareRBKTerm{ddd}{DDD}{Produit/API/Architecture}{Domain-Driven Design structurant le logiciel autour du modèle métier, des contextes délimités et d’un langage ubiquitaire.}
\DeclareRBKTerm{adr}{ADR}{Produit/API/Architecture}{Architecture Decision Record documentant une décision technique, ses options évaluées et ses impacts.}
\DeclareRBKTerm{api_gateway}{API Gateway}{Produit/API/Architecture}{Couche frontale unifiant authentification, routage, quotas et observabilité pour plusieurs services.}
\DeclareRBKTerm{service_mesh}{Service mesh}{Produit/API/Architecture}{Plan de données et de contrôle gérant découverte, sécurité, routage et observabilité entre services maillés.}
\DeclareRBKTerm{feature_flag}{Feature flag}{Produit/API/Architecture}{Interrupteur logiciel permettant d’activer ou désactiver une fonctionnalité sans redeploiement.}
\DeclareRBKTerm{canary_release}{Canary release}{Produit/API/Architecture}{Déploiement progressif sur une portion contrôlée d’utilisateurs afin de limiter l’impact en cas de régression.}
\DeclareRBKTerm{blue_green}{Blue-Green}{Produit/API/Architecture}{Stratégie de déploiement maintenant deux environnements parallèles pour basculer sans interruption de service.}
\DeclareRBKTerm{rollout_progressive}{Rollout progressif}{Produit/API/Architecture}{Montée en charge graduelle d’une nouvelle version en surveillant les indicateurs clés avant généralisation.}
\DeclareRBKTerm{ci_cd}{CI/CD}{Produit/API/Architecture}{Chaîne d’intégration et de livraison continues automatisant tests, builds, audits et déploiements.}
\DeclareRBKTerm{gitops}{GitOps}{Produit/API/Architecture}{Pratique où l’état cible de l’infrastructure et des applications est défini dans Git et appliqué automatiquement.}
\DeclareRBKTerm{mvp}{MVP}{Produit/API/Architecture}{Produit minimum viable permettant de valider une hypothèse avec l’effort minimal avant itérations.}
\DeclareRBKTerm{poc}{PoC}{Produit/API/Architecture}{Proof of Concept démontrant rapidement la faisabilité technique ou fonctionnelle d’une idée.}
\DeclareRBKTerm{rag}{RAG}{Produit/API/Architecture}{Retrieval-Augmented Generation combinant recherche documentaire et génération LLM pour produire des réponses contextualisées et vérifiables.}
\DeclareRBKTerm{llm}{LLM}{Produit/API/Architecture}{Large Language Model entraîné sur de grands corpus pour générer ou comprendre du langage naturel, intégrable via API.}
\DeclareRBKTerm{backlog}{Backlog}{Produit/API/Architecture}{File priorisée des éléments à livrer, maintenue par le Product Owner et réévaluée régulièrement.}
\DeclareRBKTerm{user_story}{User story}{Produit/API/Architecture}{Description courte d’un besoin utilisateur exprimé en langage métier avec critères d’acceptation.}
\DeclareRBKTerm{epic}{Epic}{Produit/API/Architecture}{Ensemble cohérent de user stories représentant un objectif produit significatif à découper progressivement.}
\DeclareRBKTerm{feature_toggle}{Feature toggle}{Produit/API/Architecture}{Variante de feature flag utilisée pour activer des parcours alternatifs ou des expérimentations.}
\DeclareRBKTerm{ux}{UX}{Produit}{Expérience utilisateur : qualité perçue du parcours, de la compréhension et de l’efficacité pour l’utilisateur final.}
\DeclareRBKTerm{ui}{UI}{Produit}{Interface utilisateur : éléments visuels et interactifs (écrans, composants, navigation) exposés à l’utilisateur.}
\DeclareRBKTerm{crm}{CRM}{Produit/Business}{Customer Relationship Management : outil ou processus de gestion de la relation client (prospects, ventes, support, suivi).}
\DeclareRBKTerm{readme}{README}{Ingénierie}{Fichier d’introduction d’un dépôt décrivant le projet, son installation et son usage.}
\DeclareRBKTerm{cli}{CLI}{Ingénierie}{Interface en ligne de commande permettant d’exécuter des actions et scripts via un terminal.}
\DeclareRBKTerm{latex}{LaTeX}{Documentation}{Système de composition de documents permettant de produire des PDF structurés avec une qualité typographique élevée.}
\DeclareRBKTerm{ia}{IA}{Produit/API/Architecture}{Intelligence artificielle appliquée aux produits (LLM, modèles statistiques) pour automatiser ou augmenter des parcours et décisions.}
\DeclareRBKTerm{github}{GitHub}{Produit/API/Architecture}{Plateforme de développement collaborative pour héberger code, pull requests, CI et revue entre équipes.}

% Projet / Management / Qualité
\DeclareRBKTerm{kpi}{KPI}{Projet/Management/Qualité}{Indicateur clé de performance mesurant l’atteinte d’un objectif stratégique ou opérationnel.}
\DeclareRBKTerm{okr}{OKR}{Projet/Management/Qualité}{Objectives and Key Results cadrant une ambition qualitative et des résultats mesurables à atteindre dans un délai donné.}
\DeclareRBKTerm{raci}{RACI}{Projet/Management/Qualité}{Matrice clarifiant qui est Responsable, Accountable, Consulté et Informé pour chaque activité.}
\DeclareRBKTerm{dod}{DoD}{Projet/Management/Qualité}{Definition of Done listant les critères minimaux pour considérer un élément comme terminé et livrable.}
\DeclareRBKTerm{dor}{DoR}{Projet/Management/Qualité}{Definition of Ready définissant quand une demande est suffisamment claire et estimable pour entrer en développement.}
\DeclareRBKTerm{qa}{QA}{Projet/Management/Qualité}{Quality Assurance regroupant processus et contrôles pour fiabiliser la qualité avant mise en production.}
\DeclareRBKTerm{uat}{UAT}{Projet/Management/Qualité}{User Acceptance Testing où les utilisateurs valident que la solution répond aux besoins métier.}
\DeclareRBKTerm{test_coverage}{Couverture de tests}{Projet/Management/Qualité}{Mesure de la proportion de code ou de parcours couverts par des tests automatisés ou manuels.}
\DeclareRBKTerm{shift_left}{Shift-left}{Projet/Management/Qualité}{Approche intégrant tests, sécurité et qualité dès les phases amont pour réduire coûts de correction.}
\DeclareRBKTerm{postmortem}{Postmortem}{Projet/Management/Qualité}{Analyse structurée après incident pour identifier causes racines et actions préventives sans blâme.}
\DeclareRBKTerm{retrospective}{Rétrospective}{Projet/Management/Qualité}{Rituel d’équipe inspectant le déroulement d’un cycle pour améliorer pratiques et collaboration.}
\DeclareRBKTerm{sprint}{Sprint}{Projet/Management/Qualité}{Intervalle timeboxé durant lequel une équipe livre un incrément potentiellement livrable.}
\DeclareRBKTerm{roadmap}{Roadmap}{Projet/Management/Qualité}{Plan temporel des livrables majeurs alignant vision produit, séquencement et dépendances.}
\DeclareRBKTerm{quality_gate}{Quality gate}{Projet/Management/Qualité}{Seuils de qualité automatiques à franchir avant promotion d’une version vers l’étape suivante.}
\DeclareRBKTerm{change_request}{Change request}{Projet/Management/Qualité}{Demande formelle de modification évaluée en impacts, risques et priorités avant décision.}
\DeclareRBKTerm{sign_off}{Sign-off}{Projet/Management/Qualité}{Validation formelle par les responsables attestant qu’un livrable respecte les exigences convenues.}
\DeclareRBKTerm{traceability}{Traçabilité}{Projet/Management/Qualité}{Lien bidirectionnel entre exigences, décisions, livrables et tests permettant un audit complet.}
\DeclareRBKTerm{threat_model}{Threat model}{Projet/Management/Qualité}{Analyse structurée des actifs, surfaces d’attaque et scénarios adverses afin de prioriser les contre-mesures de sécurité.}
\DeclareRBKTerm{gtm}{GTM}{Projet/Management/Qualité}{Go-To-Market : plan d’actions pour lancer un produit sur un marché cible (segmentation, offre, canaux, pricing, preuve).}
\DeclareRBKTerm{tam}{TAM}{Projet/Management/Qualité}{Total Addressable Market : taille totale théorique du marché adressable par l’offre.}
\DeclareRBKTerm{sam}{SAM}{Projet/Management/Qualité}{Serviceable Available Market : portion du TAM effectivement ciblée par l’offre.}
\DeclareRBKTerm{som}{SOM}{Projet/Management/Qualité}{Serviceable Obtainable Market : part réaliste du SAM que l’on peut capturer à court/moyen terme.}
\DeclareRBKTerm{icp}{ICP}{Projet/Management/Qualité}{Ideal Customer Profile décrivant le client idéal (secteur, taille, pain points) pour guider produit et acquisition.}
\DeclareRBKTerm{usp}{USP}{Projet/Management/Qualité}{Unique Selling Proposition : promesse différenciante qui rend l’offre préférable à celle des concurrents.}
\DeclareRBKTerm{funnel}{Funnel}{Projet/Management/Qualité}{Parcours d’acquisition décomposé en étapes (awareness → consideration → conversion) avec mesures à chaque étape.}
\DeclareRBKTerm{cohort}{Cohorte}{Projet/Management/Qualité}{Groupe d’utilisateurs entrant sur une même période, suivi pour analyser rétention et comportement.}
\DeclareRBKTerm{ab_test}{A/B test}{Projet/Management/Qualité}{Expérimentation comparant deux variantes afin de mesurer l’impact causal sur une métrique cible.}
\DeclareRBKTerm{conversion_rate}{Taux de conversion}{Projet/Management/Qualité}{Proportion d’utilisateurs qui passent à l’étape suivante du funnel ou réalisent l’action cible.}
\DeclareRBKTerm{cpi}{CPI}{Marketing/Finance}{Coût par action (inscription, installation ou événement mesuré) utilisé pour piloter l’efficacité d’une campagne.}
\DeclareRBKTerm{ko}{KO}{Projet/Qualité}{Statut indiquant l’échec d’un contrôle ou d’un gate de release, nécessitant correction avant validation.}
\DeclareRBKTerm{release_check}{release-check}{Projet/Qualité}{Commande locale standardisée qui compile le document et lance les contrôles (build + verify_params + verify_no_hardcode) avant release.}
\DeclareRBKTerm{proofs}{PROOFS}{Projet/Qualité}{Dossier de preuves (ex. PROOFS.md) regroupant les éléments vérifiables démontrant qu’un livrable respecte son Definition of Done.}
\DeclareRBKTerm{moscow}{MoSCoW}{Projet/Qualité}{Méthode de priorisation Must/Should/Could/Won’t pour cadrer le périmètre et arbitrer les fonctionnalités.}
\DeclareRBKTerm{prr}{PRR}{Projet/Qualité}{Revue préalable avant validation d’un jalon ou d’une release, vérifiant critères, risques et préparation opérationnelle.}

% Finance / Juridique / Compliance
\DeclareRBKTerm{isa}{ISA}{Finance/Juridique/Compliance}{Income Share Agreement où l’étudiant rembourse une part de revenu selon des paramètres contractuels centralisés dans la SSOT.}
\DeclareRBKTerm{ssot}{SSOT}{Finance/Juridique/Compliance}{Single Source of Truth garantissant que les paramètres contractuels et références sensibles sont définis en un point unique.}
\DeclareRBKTerm{pricing_model}{Modèle tarifaire}{Finance/Juridique/Compliance}{Structure de prix et de remises décrivant comment le service est facturé et ajusté selon les segments.}
\DeclareRBKTerm{revenue_share}{Revenue share}{Finance/Juridique/Compliance}{Accord de partage de revenus entre parties prenantes selon une clé de répartition contractuelle.}
\DeclareRBKTerm{due_diligence}{Due diligence}{Finance/Juridique/Compliance}{Examen approfondi d’une organisation ou d’un actif pour valider risques, conformité et cohérence avant engagement.}
\DeclareRBKTerm{risk_matrix}{Matrice de risques}{Finance/Juridique/Compliance}{Tableau évaluant probabilité et impact des risques pour prioriser les plans de traitement.}
\DeclareRBKTerm{compliance_by_design}{Compliance by design}{Finance/Juridique/Compliance}{Intégration anticipée des exigences réglementaires dans la conception produit et les processus.}
\DeclareRBKTerm{privacy_by_design}{Privacy by design}{Finance/Juridique/Compliance}{Inclusion proactive de la protection des données personnelles dans l’architecture et les parcours utilisateurs.}
\DeclareRBKTerm{gdpr}{RGPD}{Finance/Juridique/Compliance}{Règlement général sur la protection des données encadrant collecte, traitement et droits des personnes en Europe.}
\DeclareRBKTerm{audit_trail}{Audit trail}{Finance/Juridique/Compliance}{Journal infalsifiable retraçant actions, décisions et accès pour permettre vérification et responsabilité.}
\DeclareRBKTerm{data_retention}{Conservation des données}{Finance/Juridique/Compliance}{Politique définissant les durées et conditions de stockage, d’archivage et de suppression des données.}
\DeclareRBKTerm{segregation_of_duties}{Séparation des tâches}{Finance/Juridique/Compliance}{Principe répartissant les rôles clés pour limiter fraudes et conflits d’intérêts.}
\DeclareRBKTerm{change_control}{Change control}{Finance/Juridique/Compliance}{Processus formalisé pour évaluer, approuver et tracer toute modification sensible.}
\DeclareRBKTerm{traceability_matrix}{Matrice de traçabilité}{Finance/Juridique/Compliance}{Vue croisée reliant exigences, tests, risques et livrables pour prouver la couverture et la conformité.}
\DeclareRBKTerm{signing_authority}{Pouvoir de signature}{Finance/Juridique/Compliance}{Délégation officielle autorisant une personne à engager l’organisation par signature.}
\DeclareRBKTerm{slo}{SLO}{Finance/Juridique/Compliance}{Service Level Objective définissant une cible de qualité mesurable à atteindre sur une période donnée.}
\DeclareRBKTerm{sla}{SLA}{Finance/Juridique/Compliance}{Service Level Agreement contractuel décrivant engagements de service, métriques, pénalités et modalités de mesure.}
\DeclareRBKTerm{sli}{SLI}{Finance/Juridique/Compliance}{Service Level Indicator mesurant objectivement un aspect de la performance, base du pilotage des SLO.}
\DeclareRBKTerm{dora}{DORA}{Finance/Juridique/Compliance}{Digital Operational Resilience Act de l’UE imposant des exigences de résilience, tests et gestion des incidents pour les services financiers.}
\DeclareRBKTerm{nda}{NDA}{Finance/Juridique/Compliance}{Non-Disclosure Agreement encadrant la confidentialité et l’usage des informations échangées entre parties.}
\DeclareRBKTerm{tos}{ToS}{Finance/Juridique/Compliance}{Terms of Service définissant les conditions d’utilisation d’un service, responsabilités et limitations.}
\DeclareRBKTerm{dpa}{DPA}{Finance/Juridique/Compliance}{Data Processing Agreement précisant rôles, obligations et mesures de protection des données entre responsable et sous-traitant.}
\DeclareRBKTerm{pii}{PII}{Finance/Juridique/Compliance}{Personally Identifiable Information : données permettant d’identifier une personne et nécessitant une protection renforcée.}
\DeclareRBKTerm{cac}{CAC}{Finance/Juridique/Compliance}{Customer Acquisition Cost : coût moyen pour acquérir un client, incluant marketing et ventes.}
\DeclareRBKTerm{ltv}{LTV}{Finance/Juridique/Compliance}{Lifetime Value : revenu estimé (après coûts de service) généré par un client sur toute sa durée de vie.}
\DeclareRBKTerm{arpu}{ARPU}{Finance/Juridique/Compliance}{Average Revenue Per User : revenu moyen par utilisateur sur une période donnée.}
\DeclareRBKTerm{mrr}{MRR}{Finance/Juridique/Compliance}{Monthly Recurring Revenue : revenu mensuel récurrent, base de pilotage SaaS.}
\DeclareRBKTerm{arr}{ARR}{Finance/Juridique/Compliance}{Annual Recurring Revenue : revenu récurrent annualisé, indicateur de taille et de croissance.}
\DeclareRBKTerm{churn}{Churn}{Finance/Juridique/Compliance}{Taux de perte de clients ou d’utilisateurs sur une période, mesurant l’attrition.}
\DeclareRBKTerm{retention}{Rétention}{Finance/Juridique/Compliance}{Capacité à conserver clients ou utilisateurs sur la durée, inverse du churn.}
\DeclareRBKTerm{roi}{ROI}{Finance/Juridique/Compliance}{Return on Investment : rapport entre gain généré et investissement engagé.}
\DeclareRBKTerm{roas}{ROAS}{Finance/Juridique/Compliance}{Return On Ad Spend : revenu généré divisé par la dépense publicitaire correspondante.}
\DeclareRBKTerm{cashflow}{Cashflow}{Finance/Juridique/Compliance}{Flux de trésorerie net sur une période, mesurant entrées et sorties de cash.}
\DeclareRBKTerm{burn_rate}{Burn rate}{Finance/Juridique/Compliance}{Vitesse à laquelle la trésorerie est consommée chaque mois (cash out net).}
\DeclareRBKTerm{runway}{Runway}{Finance/Juridique/Compliance}{Nombre de mois de trésorerie restant compte tenu du burn rate actuel.}
\DeclareRBKTerm{unit_economics}{Unit economics}{Finance/Juridique/Compliance}{Analyse des revenus et coûts par unité (client/étudiant) pour mesurer viabilité et marge contributive.}
\DeclareRBKTerm{break_even}{Break-even}{Finance/Juridique/Compliance}{Point d’équilibre où les revenus couvrent l’ensemble des coûts, sans perte ni profit.}
\DeclareRBKTerm{tnd}{TND}{Finance/Juridique/Compliance}{Dinar tunisien, devise utilisée pour exprimer les montants dans l’offre et les projections financières du programme.}
\DeclareRBKTerm{usd}{USD}{Finance/Juridique/Compliance}{Dollar américain : devise de référence utilisée pour exprimer des coûts, revenus ou comparaisons internationales.}
\DeclareRBKTerm{ca}{CA}{Finance/Juridique/Compliance}{Chiffre d’affaires : montant total des ventes sur une période donnée, avant charges et impôts.}
\DeclareRBKTerm{rbk}{RBK}{Organisation}{Organisation commanditaire et destinataire du présent livre blanc, responsable de la validation du cadre et du déploiement du programme.}
\DeclareRBKTerm{ceo}{CEO}{Organisation}{Chief Executive Officer dirigeant l’organisation, arbitre les priorités et alloue les ressources pour atteindre les objectifs stratégiques.}
\DeclareRBKTerm{pr}{PR}{Organisation}{Relations publiques assurant la communication institutionnelle et la gestion de l’image auprès des médias et parties prenantes.}
\DeclareRBKTerm{lms}{LMS}{Éducation}{Learning Management System : plateforme de gestion de contenus pédagogiques, d’activités, d’évaluations et de suivi des apprenants.}
\DeclareRBKTerm{latex}{LaTeX}{Documentation}{Système de composition de documents permettant de produire des PDF structurés avec une qualité typographique élevée.}
