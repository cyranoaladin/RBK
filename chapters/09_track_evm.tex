\chapter[Track B : EVM Engineer]{TRACK B : EVM ENGINEER (SOLIDITY/FOUNDRY)}

\section{Philosophie du Track : La Maîtrise du Standard Industriel}

L'\RBKTerm{evm} (Ethereum Virtual Machine) est le standard mondial des smart contracts. Maîtriser Solidity et Foundry, c'est s'ouvrir les portes de l'écosystème le plus riche (Ethereum, Arbitrum, Optimism, Base, Polygon).
Notre approche est "Infrastructure-First". Nous ne formons pas des développeurs qui copient-collent du code OpenZeppelin, mais des ingénieurs capables de comprendre le stockage bas niveau, l'optimisation du Gas et les subtilités des upgrades (Proxies).

\paragraph{Positionnement "Infra Engineer"}
L'ingénieur \RBKTerm{evm} RBK est un bâtisseur de protocoles. Il maîtrise la chaîne DevOps (Foundry, \RBKTerm[CI]{ci_cd}, Verification), la sécurité offensive (Fuzzing, Invariants) et les patterns de composabilité (\RBKTerm{defi} Lego).

\begin{tcolorbox}[colback=blue!5!white,colframe=blue!75!black,title=NON NÉGOCIABLE : AUDIT-READINESS]
\begin{itemize}
    \item \textbf{Test-First :} TDD strict avec Foundry. Fuzzing obligatoire.
    \item \textbf{Gas Optimization :} Chaque Opcode compte (Assembly si nécessaire).
    \item \textbf{Security Mindset :} "Don't trust, verify". Protection Reentrancy, Overflow, Access Control.
\end{itemize}
\end{tcolorbox}

\begin{figure}[H]
    \centering
    \begin{tikzpicture}
        \node[draw, rectangle] (sol) {Solidity};
        \node[draw, rectangle, right=of sol] (test) {Tests (Foundry)};
        \node[draw, rectangle, right=of test] (sec) {Security (Slither)};
        \node[draw, rectangle, below=of test] (deploy) {Deploy (Scripts)};
        \draw[->] (sol) -- (test);
        \draw[->] (test) -- (sec);
        \draw[->] (sec) -- (deploy);
    \end{tikzpicture}
    \caption{Chaîne de Valeur EVM}
    \label{fig:trackB_valuechain}
\end{figure}

\section{Structure Pédagogique : De la Logique au Durcissement (16 Semaines)}

Un parcours intensif qui commence par les fondations (Storage Layout) pour aller jusqu'au déploiement multi-chain complexe.

\begin{table}[H]
    \caption{Carte des Modules Track B}
    \centering
    \small
    \begin{tabularx}{\textwidth}{|l|l|X|l|}
        \hline
        \textbf{Module} & \textbf{Sem} & \textbf{Objectif} & \textbf{Livrable} \\ \hline
        1. Basics & 13-15 & EVM Internals & Vault Natif \\ \hline
        2. Pro Env & 16-18 & Foundry Mastery & CI Pipeline \\ \hline
        3. Standards & 19-21 & ERC20/721 & Token System \\ \hline
        4. dApp & 22-24 & Intégration & Full Stack dApp \\ \hline
        5. Scaling & 25-26 & L2 \& Upgrades & UUPS Proxy \\ \hline
        6. Hardening & 27-28 & Sécurité & Audit Report \\ \hline
    \end{tabularx}
\end{table}

\subsection{MODULE 1 : Smart Contract Basics \& Solidity Deep Dive (Semaines 13-15)}

\paragraph{Objectifs}
Comprendre comment l'EVM stocke les données (Slots), la différence Memory/CallData, et les structures de contrôle de base.
\textbf{Lab A (Vault Sécurisé)} : Un contrat de dépôt/retrait avec gestion des rôles (Ownable).
\textbf{Critères :} Tests de cas nominaux et d'erreurs (Revert).

\subsection{MODULE 2 : Environnement de Développement Pro (Semaines 16-18)}

\paragraph{Objectifs}
Passer de Remix à Foundry. Maîtriser `forge test`, `cast`, et le Fuzzing.
\textbf{Lab (Test Suite)} : Écrire une suite de tests exhaustive (Unit + Fuzz) pour un protocole existant (ex: Uniswap V2 Pair simplifié).
\textbf{Livrable :} Repo avec GitHub Actions qui lance les tests à chaque PR.

\subsection{MODULE 3 : Token Standards \& Composabilité (Semaines 19-21)}

\paragraph{Objectifs}
Implémenter ERC20, ERC721, ERC1155. Comprendre `approve`, `transferFrom` et les risques associés.
\textbf{Lab (\RBKTerm{defi} Lego)} : Un contrat qui "wrap" un token pour ajouter du rendement (\RBKTerm{staking}).
\textbf{Critères :} Interopérabilité vérifiée avec les standards.

\subsection{MODULE 4 : dApp Development \& Web3 Integration (Semaines 22-24)}

\paragraph{Objectifs}
Connecter un Front (React/Next) au contrat via Wagmi/Viem. Gérer le cycle de vie de la transaction (Pending, Confirmed, Failed).
\textbf{Lab (Mini-DEX UI)} : Interface pour swapper des tokens (simulation).
\textbf{Checklist :} Gestion des erreurs \RBKTerm{rpc}, Feedback utilisateur.

\subsection{MODULE 5 : L2 Scaling \& Advanced Patterns (Semaines 25-26)}

\paragraph{Objectifs}
Comprendre les Rollups (Optimistic/ZK). Déployer sur Arbitrum/Base. Gérer l'upgradeabilité (Proxies).
\textbf{Lab (UUPS Upgrade)} : Déployer une V1, puis upgrader vers une V2 sans perdre l'état (Storage).

\subsection{MODULE 6 : Production Hardening \& Security (Semaines 27-28)}

\paragraph{Objectifs}
Sécurisation finale. Audit interne.
\textbf{Projet Final :} Déploiement d'un protocole complet sur Testnet (Sepolia/Goerli) avec scripts de vérification Etherscan automatisés.

\begin{table}[H]
    \caption{Security Checklist EVM}
    \centering
    \small
    \begin{tabularx}{\textwidth}{|l|X|l|}
        \hline
        \textbf{Vulnérabilité} & \textbf{Contrôle} & \textbf{Outil} \\ \hline
        Reentrancy & Checks-Effects-Interactions & Slither \\ \hline
        Access Control & Modifiers corrects & Manual Review \\ \hline
        Arithmetic & Overflow (Solidity <0.8 checked) & Fuzzing \\ \hline
    \end{tabularx}
\end{table}

\section{Stack Technique Spécifique}

On privilégie la stack moderne (Rust-based) pour sa rapidité.

\begin{table}[H]
    \caption{Stack Track B (Foundry)}
    \centering
    \small
    \begin{tabularx}{\textwidth}{|l|X|l|}
        \hline
        \textbf{Catégorie} & \textbf{Outils} & \textbf{Usage} \\ \hline
        Framework & Foundry (Forge, Cast, Anvil) & Tout-en-un \\ \hline
        Libs & OpenZeppelin Contracts & Standards sécu \\ \hline
        Client & Viem, Wagmi & Front-end \\ \hline
        Analysis & Slither, Aderyn & Static Analysis \\ \hline
    \end{tabularx}
\end{table}

\section{Profil de Sortie : L'Ingénieur d'Infrastructure EVM}

L'Ingénieur EVM RBK est prêt pour intégrer une équipe Core Protocol ou une start-up \RBKTerm{defi}. Il sait écrire du code qui gère des millions de dollars.

\paragraph{Missions Types en Entreprise}
\begin{itemize}
    \item Développer une stratégie de Loop \RBKTerm{staking} sur un protocole de Lending (ex: Aave).
    \item Migrer un token Gouvernance d'un Layer 1 vers un Layer 2 (\RBKTerm{bridge} Architecture).
    \item Écrire les scripts de déploiement automatisés pour une collection NFT de 10k pièces avec whitelist Merkle Tree.
\end{itemize}

La trajectoire comprend un volet \RBKTerm[Ops]{ops} pour fiabiliser les déploiements et vérifier les contrats.

\begin{table}[H]
    \caption{Matrice Compétences Infra EVM}
    \centering
    \small
    \begin{tabularx}{\textwidth}{|l|X|l|}
        \hline
        \textbf{Domaine} & \textbf{Attendu} & \textbf{Preuve} \\ \hline
        Solidity & Expert (Storage, Assembly) & Gas Golfing Repo \\ \hline
        Testing & Expert (Fuzzing, Invariants) & Coverage > 95\% \\ \hline
        Ops & Autonome (Scripts, Verify) & Déploiements vérifiés \\ \hline
    \end{tabularx}
\end{table}
