\chapter{CAPSTONES (PROJETS SIGNATURES)}

\section{Philosophie du Capstone : Le Standard « Studio »}

Chez RBK, un Capstone n'est pas un projet d'école. C'est un produit "Mainnet-Ready" qui respecte les standards d'un studio de développement professionnel.
Il ne s'agit pas de prouver que "ça marche", mais que "ça ne peut pas casser".

\begin{table}[H]
    \caption{Studio-Grade Checklist (Non-Négociable)}
    \centering
    \small
    \begin{tabular}{|l|p{8cm}|l|}
        \hline
        \textbf{Catégorie} & \textbf{Exigence} & \textbf{Preuve} \\ \hline
        Sécurité & Threat Model formalisé AVANT le code & Doc STRIDE \\ \hline
        Qualité & Zéro warning linter, coverage > 80\% & Rapport CI \\ \hline
        Ops & Déploiement scripté et reproductible & Makefile / Taskfile \\ \hline
    \end{tabular}
\end{table}

\section{Les 3 Projets Signatures (Cahier des Charges Industriel)}
Chaque Capstone simule un livrable client. Le niveau d'exigence est "Audit Ready".

\subsection{Capstone 1 — Wallet \& Transaction Reliability Pack (Frontend/UX)}
\textbf{Focus :} Expérience Utilisateur \& Fiabilité RPC.

\subsubsection{Spécifications Fonctionnelles (User Stories)}
\begin{enumerate}
    \item \textbf{US.1 (Feedback) :} "En tant qu'utilisateur, je veux voir un feedback précis (Spinner, Toast) pour chaque état de transaction (Signing, Sending, Confirming, Finalized)."
    \item \textbf{US.2 (Retry) :} "En tant qu'utilisateur, si le RPC échoue, je veux que l'app réessaie automatiquement (Backoff exponentiel) sans que je reclique."
    \item \textbf{US.3 (history) :} "En tant que dev, je veux un log persistant des transactions échouées pour debugger."
\end{enumerate}

\subsubsection{Spécifications Techniques}
\begin{itemize}
    \item \textbf{Stack :} React (Next.js), TanStack Query, Viem (EVM) ou Solana Web3.js.
    \item \textbf{State Machine :} Implémentation stricte (Idle $\to$ Signing $\to$ Broadcasting $\to$ Confirming).
    \item \textbf{Failover :} Configuration d'au moins 2 RPCs (Primaire + Fallback).
\end{itemize}

\subsubsection{Checklist Sécurité \& Qualité}
\begin{itemize}
    \item [ ] Pas de Private Key stockée (Wallet Adapter Only).
    \item [ ] Sanitize des inputs utilisateurs (Adresses, Montants).
    \item [ ] Gestion du "Slippage" (Tolérance affichée).
\end{itemize}

\subsubsection{Critères d'Acceptation \& CI}
\begin{itemize}
    \item \textbf{Test :} Simulation de coupure réseau (Mock RPC) $\to$ le Retry fonctionne.
    \item \textbf{CI :} Linting strict (ESLint), Prettier, Build success.
    \item \textbf{Doc :} README avec diagramme de séquence (Mermaid).
\end{itemize}


\subsection{Capstone 2 — Tokenization \& Admin Control Center (RWA/DeFi)}
\textbf{Focus :} Gestion des droits (RBAC) et Logique Métier.

\subsubsection{Spécifications Fonctionnelles}
\begin{enumerate}
    \item \textbf{US.1 (Role) :} "En tant que SuperAdmin, je peux nommer ou révoquer un 'Operator'."
    \item \textbf{US.2 (Freeze) :} "En tant qu'Operator, je peux geler un compte utilisateur suspect (Compliance)."
    \item \textbf{US.3 (Audit) :} "En tant qu'Auditeur, je peux consulter l'historique on-chain de toutes les actions Admin."
\end{enumerate}

\subsubsection{Spécifications Techniques}
\begin{itemize}
    \item \textbf{Stack :} Solidity (OZ AccessControl) ou Anchor (PDA Authorities).
    \item \textbf{Invariants :} TotalSupply doit toujours égaler la somme des balances (sauf mint/burn explicite).
    \item \textbf{Events :} Émission d'événement pour CHAQUE changement d'état critique.
\end{itemize}

\subsubsection{Checklist Sécurité Dédiée}
\begin{itemize}
    \item [ ] \textbf{Access Control :} Vérifier \texttt{onlyRole} sur toutes les fonctions sensibles.
    \item [ ] \textbf{Two-Step Transfer :} Changement d'admin en 2 étapes (Propose + Accept).
    \item [ ] \textbf{Checks-Effects-Interactions :} Respect strict du pattern anti-reentrancy.
\end{itemize}


\subsection{Capstone 3 — Digital Assets \& Utility Ecosystem (NFT/Gaming)}
\textbf{Focus :} Performance \& Vérification de propriété.

\subsubsection{Spécifications Fonctionnelles}
\begin{enumerate}
    \item \textbf{US.1 (Gating) :} "En tant qu'utilisateur, je ne peux accéder au contenu Beta que si je possède le NFT 'Early Bird'."
    \item \textbf{US.2 (Claim) :} "En tant que joueur, je peux clamer mon reward une seule fois par jour."
\end{enumerate}

\subsubsection{Spécifications Techniques}
\begin{itemize}
    \item \textbf{Mécanisme :} Merkle Tree (Allowlist) ou Signature verifier backend.
    \item \textbf{Optimisation :} Utilisation de la compression (cNFT Solana) ou ERC-721A (EVM) pour réduire les coûts de mint.
\end{itemize}

\subsubsection{Checklist Sécurité Dédiée}
\begin{itemize}
    \item [ ] \textbf{Bot Protection :} Captcha ou signature nonce pour empêcher le mint massif par bots.
    \item [ ] \textbf{Replay Attack :} Vérification que la signature n'a pas déjà été utilisée.
    \item [ ] \textbf{Metadata :} Fichiers hébergés sur IPFS (pas de liens HTTP centralisés).
\end{itemize}

\section{Matrice d'Évaluation (Rubric Studio)}
Chaque projet est noté sur 100 points selon une grille industrielle stricte.

\begin{table}[H]
    \caption{Rubric d'Évaluation Studio (100 pts)}
    \centering
    \small
    \rowcolors{2}{gray!5}{white}
    \begin{tabularx}{\textwidth}{|l|c|X|l|}
        \hline
        \textbf{Pilier} & \textbf{Pts} & \textbf{Critères Évalués} & \textbf{Fail Condition} \\ \hline
        \textbf{Engineering} & \textbf{40} & Architecture propre, Code modulaire, Gestion d'erreurs (Option/Result), Choix de structures de données. & Code monolithique, "Unwrap" sauvages. \\ \hline
        \textbf{Sécurité} & \textbf{30} & Respect du Threat Model, Validation des inputs, Access Control, Tests de cas limites (Fuzzing basic). & Vulnérabilité critique, Clé privée dans repo. \\ \hline
        \textbf{Delivery} & \textbf{20} & CI/CD fonctionnel, Git Flow propre (Branches, Commits), README complet, Déploiement reproductible. & Pas de doc, Build cassé. \\ \hline
        \textbf{Comms} & \textbf{10} & Vidéo démo claire, Pitch deck concis, Capacité à expliquer les choix techniques (Trade-offs). & Démo illisible, Incapable de justifier un choix. \\ \hline
    \end{tabularx}
\end{table}

\section{Délivrables de Sortie (Le "Package")}

Chaque étudiant doit remettre un "Package" zippé (ou repo public) contenant :
1. \textbf{Le Code :} Clean, commenté, testé.
2. \textbf{La Documentation :} Architecture, Setup, API.
3. \textbf{Le Rapport d'Auto-Audit :} "J'ai cherché à me hacker, voici ce que j'ai trouvé".
4. \textbf{La Vidéo Démo :} 3 minutes max, scénarisée.

Ce package est votre passeport pour l'emploi. Il remplace le CV.

\begin{figure}[H]
    \centering
    \begin{tikzpicture}
        \node[draw] (code) {Code};
        \node[draw, right=of code] (test) {Test};
        \node[draw, right=of test] (audit) {Audit Doc};
        \node[draw, right=of audit] (release) {Release};
        \node[draw, right=of release] (demo) {Demo};
        \draw[->] (code) -- (test);
        \draw[->] (test) -- (audit);
        \draw[->] (audit) -- (release);
        \draw[->] (release) -- (demo);
    \end{tikzpicture}
    \caption{Pipeline Packaging}
\end{figure}
