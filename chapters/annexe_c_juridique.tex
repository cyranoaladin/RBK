\chapter{ANNEXE — CADRE JURIDIQUE \&\\ CONFORMITÉ (TUNISIE)}
\label{ann:juridique}

\section{Synthèse Juridique : Opérer depuis la Tunisie}
Grâce au statut Entreprise Totalement Exportatrice (ETE), l'ingénieur RBK bénéficie d'une exonération fiscale massive sur ses revenus étrangers (0\% IS pendant 4 ans, puis 10\%). Ce cadre, couplé à une gestion rigoureuse des flux crypto/fiat, fait de la Tunisie un hub Web3 ultra-compétitif.

\section{Statut d'Entreprise Totalement\\ Exportatrice (ETE)}

Nous rappelons ici le cadre légal tunisien qui permet d'opérer en devises tout en restant conforme.

\subsection{Définition et Cadre Légal}
L'Entreprise Totalement Exportatrice (ETE) est un régime fiscal tunisien réglementé par le Code d'Incitation aux Investissements (Loi n°2016-71) et le Décret n°2017-758. Il permet aux entités réalisant 100\% de leur chiffre d'affaires à l'export de bénéficier d'avantages majeurs.

\section{Avantages Fiscaux}
\begin{itemize}
	\item \textbf{Impôts Sociétés (IS) :} Exonération totale pendant 4 ans, puis taux réduit à 10\% (vs 15\% standard).
	\item \textbf{Confidentialité B2B :} Pour tout contrat international, signer un \RBKTerm{nda} précisant périmètre et durée.
	\item \textbf{Devises :} Liberté totale de gestion des comptes en devises étrangères (EUR/\RBKTerm{usd}) sans autorisation préalable de la BCT pour les opérations liées à l'activité.
	\item \textbf{TVA :} Exonération de TVA sur les services et biens acquis pour l'exportation (en suspension de taxes).
	\item \textbf{Dividendes :} Exonération de retenue à la source sur les dividendes distribués.
\end{itemize}

\subsection{Conditions d'Éligibilité pour RBK 2.0}
Pour bénéficier du statut ETE, RBK (et ses alumni entrepreneurs) doit :
\begin{itemize}
	\item Exporter 100\% de ses services à l'étranger (formation remote, consulting, audit).
	\item Justifier d'un plan d'affaires et créer un minimum d'emplois.
\end{itemize}

\subsection{Avantages Fiscaux Comparés}
\begin{table}[H]
	\caption{Comparatif Fiscal : Standard vs ETE}
	\centering
	\begin{tabularx}{\textwidth}{|l|l|Y|}
		\hline
		\textbf{Indicateur} & \textbf{Régime Standard} & \textbf{Régime ETE (Export)}   \\ \hline
		IS (Impôt Sociétés) & 15\% dès Année 1         & 0\% (4 ans) puis 15\%          \\ \hline
		Dividendes          & Retenue à la source 10\% & Exonérés (si bénéfices export) \\ \hline
		TVA Achats          & 19\%                     & Suspension de TVA              \\ \hline
		Compte Bancaire     & TND uniquement           & Devises + TND                  \\ \hline
	\end{tabularx}
\end{table}

\section{Kit de Survie Juridique Freelance}

Cette section offre des réflexes juridiques rapides pour sécuriser les missions internationales.

\subsection{Matrice de Décision : Patente vs SUARL}
\begin{table}[H]
	\centering
	\begin{tabularx}{\textwidth}{|l|Y|Y|}
		\hline
		\textbf{Critère} & \textbf{Patente (Pers. Physique)} & \textbf{SUARL (Pers. Morale)} \\ \hline
		Coût Création    & Quasi-nul                         & Moyen (1000 TND + Capital)    \\ \hline
		Complexité       & Très faible                       & Moyenne                       \\ \hline
		Responsabilité   & Illimitée                         & Limitée au capital            \\ \hline
		Recommandation   & Pour débuter (< 50k TND/an)       & Dès que \RBKTerm{ca} > 80k TND/an       \\ \hline
	\end{tabularx}
\end{table}

\subsection{Checklist Création d'Entreprise ETE}
\begin{enumerate}
	\item[{\faIcon[regular]{square}}] J-0 : Rédaction des statuts (Objet social : "Export de services informatiques").
	\item[{\faIcon[regular]{square}}] J-2 : Dépôt dossier APII en ligne (Déclaration d'investissement).
	\item[{\faIcon[regular]{square}}] J-15 : Obtention de l'attestation de dépôt APII.
	\item[{\faIcon[regular]{square}}] J-20 : Enregistrement Recette Finance (Timbre fiscal).
	\item[{\faIcon[regular]{square}}] J-30 : Immatriculation RNE (Registre National des Entreprises).
	\item[{\faIcon[regular]{square}}] J-35 : Ouverture Compte Bancaire "Dossier Juridique" (+ Compte Devises).
\end{enumerate}

\section{Mécanisme de Paiement\\ \texorpdfstring{Crypto $\to$ Fiat}{Crypto -> Fiat} Conforme}

Nous décrivons le chemin de paiement recommandé pour rester traçable et compatible avec les exigences bancaires.

\subsection{Traçabilité Comptable}
Pour chaque transaction entrante :
\begin{enumerate}
	\item Émettre une facture en Devises (EUR/\RBKTerm{usd}) mentionnant "Règlement par voie électronique".
	\item Conserver le "Transaction Hash" comme preuve d'exécution.
	\item Obtenir l'avis de crédit bancaire mentionnant l'origine des fonds (Bitwage/Grey).
	\item Comptabiliser en TND au taux du jour de réception.
\end{enumerate}

\section{Validation Juridique des ISA}

Ce passage précise les garde-fous qui rendent les ISA applicables dans le contexte tunisien.

\subsection{Qualification Juridique (COC)}
Le contrat ISA est qualifié de Contrat Innommé (Article 2 du Code des Obligations et Contrats), régi par la volonté des parties tant qu'il ne contrevient pas à l'ordre public. Il s'apparente à :
\begin{itemize}
	\item Un Prêt à Rémunération Variable.
	\item Un contrat de Musharaka (Finance islamique).
\end{itemize}

\subsection{Risques Juridiques \& Mitigation}
\begin{table}[H]
	\caption{Matrice des Risques Principaux}
	\centering
	\begin{tabularx}{\textwidth}{|l|l|l|Y|}
		\hline
		\textbf{Risque}     & \textbf{Prob.} & \textbf{Imp.} & \textbf{Mitigation}                              \\ \hline
		Requalification ISA & Moy.           & Élevé         & Cap à 1.5x, durée limitée, validation avocat.    \\ \hline
		Blocage Crypto      & Faible         & Critique      & Alternative TND + Structure Offshore de secours. \\ \hline
	\end{tabularx}
\end{table}

\section{Plan de Continuité Juridique}
\begin{itemize}
	\item \textbf{Scénario 1 : Changement réglementaire défavorable.} Action : Bascule 100\% TND via partenaires bancaires locaux. Migration de l'entité légale IP à l'étranger.
	\item \textbf{Scénario 2 : Défaut massif ISA (>30\%).} Action : Activation du Fonds de Garantie (50k TND). Restructuration des dettes.
\end{itemize}
