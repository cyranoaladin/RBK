\chapter[Business Plan]{BUSINESS PLAN \& STRATÉGIE DE CROISSANCE}
\label{chap:business_plan}

\section{Modèle Économique Hybride}

RBK 2.0 repose sur une diversification des sources de revenus pour garantir sa pérennité indépendamment des cycles du marché crypto.

\begin{figure}[H]
    \centering
    \begin{tikzpicture}
        \draw[fill=SolanaPurple!20] (0,0) circle (2cm);
        \node at (0,0.5) {\textbf{B2C (70\%)}};
        \node at (0,-0.5) {\footnotesize Formation Initiale};
        
        \draw[fill=SolanaBlue!20] (3,0) circle (1.5cm);
        \node at (3,0.5) {\textbf{ISA (20\%)}};
        \node at (3,-0.5) {\footnotesize Success Fees};
        
        \draw[fill=SolanaGreen!20] (5.5,0) circle (1cm);
        \node at (5.5,0.3) {\textbf{B2B (10\%)}};
        \node at (5.5,-0.3) {\footnotesize Corporate};
    \end{tikzpicture}
    \caption{Mix Revenus Cible (Année 3)}
\end{figure}

\section{Hypothèses & Sources du Modèle}

Ce modèle repose sur des données de marché comparables (Bootcamps Web3, Market Research 2024-2025).

\subsection{Hypothèses Clés}
\begin{itemize}
    \item \textbf{CAC (Coût d'Acquisition Client)} : Estimé à \textbf{250 TND / étudiant converti} (Campagnes LinkedIn/FB ciblées + SEO).
    \item \textbf{Salaire de Sortie Moyen} : Conservateur à \textbf{2 500 TND Net/Mois} (Marché local Senior) ou \textbf{1 500 \$} (Remote Junior Global).
    \item \textbf{Taux de Chute (Churn)} : 15\% entre chaque niveau (filtrage de qualité).
    \item \textbf{Taux de Placement} : 80\% à 6 mois (standard industrie pour formations sélectives).
\end{itemize}

\subsection{Structure des Coûts Directs}
\begin{itemize}
    \item \textbf{Mentorat} : Ratio 1:10. Rémunération horaire ou forfaitaire (Variable).
    \item \textbf{Infra} : Coût fixe LMS + Cloud (AWS/RPC) = 5\% du CA.
    \item \textbf{Légal/Admin} : 5\% du CA.
\end{itemize}

\section{Funnel d'Acquisition & Sourcing}

Le modèle de rentabilité dépend de la sélectivité en entrée (qualité des profils = placement garanti).

\begin{enumerate}
    \item \textbf{Top of Funnel (Leads)} : 1000 Inscrits Webinaire / Downloads Livre Blanc.
    \item \textbf{Middle of Funnel (Applicants)} : 200 Candidats passent le test technique Python/JS.
    \item \textbf{Bottom of Funnel (Selection)} : 50 Admissibles après entretien.
    \item \textbf{Conversion (Cohorte)} : 30 Inscrits payants (ou boursiers validés).
\end{enumerate}

Le ratio cible est de \textbf{3\% de conversion Lead $\to$ Client}, assurant une densité de talents élevée.

\section{Le Pilier B2B : Corporate Upskilling}

Pour réduire la dépendance aux frais de scolarité individuels, nous lançons une offre dédiée aux entreprises (Banques, ESN, Telcos) souhaitant monter une "Blockchain Factory" interne.

\textbf{Offre "Corporate Cohort" :}
\begin{itemize}
    \item \textbf{Principe :} Une entreprise réserve un bloc de 3 à 5 sièges dans une cohorte pour ses employés.
    \item \textbf{Tarif :} \textbf{15 900 TND / siège} (Premium Pricing).
    \item \textbf{Avantages :} Suivi RH dédié, Capstone orienté sur un Use-Case de l'entreprise, Clause de confidentialité.
    \item \textbf{Objectif :} 30\% du CA total en Année 3.
\end{itemize}

\section{Trajectoire Financière (36 Mois)}

La trésorerie est le nerf de la guerre. Notre modèle prend en compte le "Cash Drag" (décalage) des ISA.

\textbf{Projection du Volume Étudiant (Funnel) :}
\begin{center}
\small
\begin{tabular}{|l|c|c|c|}
\hline
\textbf{Niveau} & \textbf{Année 1} & \textbf{Année 2} & \textbf{Année 3} \\ \hline
\textbf{Niveau 1} (Foundations) & 30 & 60 & 90 \\ \hline
\textbf{Niveau 2} (Builder) & 20 & 40 & 60 \\ \hline
\textbf{Niveau 3} (Pro/Audit) & 12 & 24 & 36 \\ \hline
\end{tabular}
\end{center}

\begin{table}[H]
    \caption{Compte de Résultat et Trésorerie Prévisionnelle (TND)}
    \centering
    \small
    \rowcolors{2}{gray!5}{white}
    \begin{tabularx}{\textwidth}{|l|r|r|r|}
        \hline
        \textbf{Indicateur} & \textbf{Année 1 (Amorçage)} & \textbf{Année 2 (Scale)} & \textbf{Année 3 (Maturité)} \\ \hline
        \textbf{CA FORMATION (L1+L2+L3)} & \textbf{311 800} & \textbf{623 600} & \textbf{935 400} \\ \hline
        \textbf{CA ISA (Différé)} & 0 & 120 000 & 350 000 \\ \hline
        \textbf{CA SERVICES (B2B)} & 20 000 & 80 000 & 200 000 \\ \hline
        \textbf{TOTAL REVENUS} & \textbf{331 800} & \textbf{823 600} & \textbf{1 485 400} \\ \hline
        \textbf{DÉPENSES (OPEX)} & \textbf{260 000} & \textbf{520 000} & \textbf{880 000} \\ \hline
        \textit{(dont Salaires Staff)} & \textit{150 000} & \textit{300 000} & \textit{500 000} \\ \hline
        \textit{(dont Mentors Variable)} & \textit{60 000} & \textit{120 000} & \textit{200 000} \\ \hline
        \textbf{EBITDA} & \textbf{+71 800} & \textbf{+303 600} & \textbf{+605 400} \\ \hline
        \textit{Marge \%} & \textit{21\%} & \textit{37\%} & \textit{41\%} \\ \hline
    \end{tabularx}
\end{table}

\textbf{Analyse de Trésorerie :}
L'Année 1 est financée par le mix Upfront. L'effet de levier ISA commence à impacter significativement la trésorerie au milieu de l'Année 2, créant un "Fond de Roulement" naturel pour l'expansion.

\section{Analyse de Sensibilité}

Nous stress-testons le modèle selon 3 variables critiques.

\begin{table}[H]
    \caption{Sensibilité de l'EBITDA Année 2 (Objectif Cible 303k)}
    \centering
    \small
    \begin{tabular}{|l|c|c|l|}
        \hline
        \textbf{Variable} & \textbf{Variation} & \textbf{Impact EBITDA} & \textbf{Commentaire} \\ \hline
        Taux Placement & -20 pts & -50k & Impact différé sur ISA. \\ \hline
        Prix Bundle & -20\% & -120k & \textbf{Critique}. Nécessite réduction coûts Mentors. \\ \hline
        Conv. N1$\to$N2 & -15 pts & -90k & Critique. Nécessite meilleur sourcing S0. \\ \hline
    \end{tabular}
\end{table}

\subsection{Gestion du Risque Crédit ISA}
L'ISA est un actif financier qui comporte des risques spécifiques.
\begin{enumerate}
    \item \textbf{Recouvrement :} Nous intégrons une hypothèse de "Défaut Technique" de 15\% (étudiants ne payant pas malgré un emploi).
    \item \textbf{Mitigation :}
    \begin{itemize}
        \item \textbf{Juridique :} Contrat enregistré avec reconnaissance de dette.
        \item \textbf{Reputation :} Le SBT (Diplôme) est révocable ou marqué "En défaut" on-chain en cas d'impayé avéré, bloquant l'accès au réseau Alumni.
        \item \textbf{Incitations :} Bonus de fin de contrat si paiement anticipé.
    \end{itemize}
\end{enumerate}

\section{Financements et Partenariats Stratégiques}

Pour accélérer sans diluer le capital, RBK active les leviers non-dilutifs :

\subsection{1. Écosystème Web3 (Grants)}
\begin{itemize}
    \item \textbf{Solana Foundation :} Demande de grant "Education" pour financer les serveurs et les bourses (Target: 50k\$).
    \item \textbf{Superteam :} Sponsoring des Hackathons de fin de cohorte (Prize pool).
\end{itemize}

\subsection{2. Bailleurs de Fonds Institutionnels}
\begin{itemize}
    \item \textbf{Union Européenne (Erasmus+ / Horizon Europe) :} Projets de mobilité des talents numériques Afrique-Europe.
    \item \textbf{Banque Africaine de Développement (BAD) :} Programme "Coding for Jobs".
\end{itemize}

\subsection{3. Modèle de Franchise (Scale Africa)}
Dès l'Année 3, le modèle "RBK in a Box" (LMS + Programme + Brand) sera proposé en franchise à des hubs technologiques au Sénégal et en Côte d'Ivoire.
\begin{itemize}
    \item \textbf{Modèle :} Revenue Share (20\% du CA Franchise).
    \item \textbf{Apport :} RBK fournit la plateforme et la certification SBT. Le partenaire gère le local et le sourcing.
\end{itemize}
