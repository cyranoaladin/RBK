\chapter{Annexes — Gabarits opérationnels \& Stratégiques}
\label{annexe:gabarits}

Cette annexe présente les matrices stratégiques validées pour le déploiement de RBK 2.0.

\paragraph{Légende des rôles (RACI).}
\begin{itemize}[leftmargin=*]
  \item \textbf{CEO} : Direction RBK (décisions stratégiques / arbitrage).
  \item \textbf{HoP} : Head of Program / Directeur pédagogique (qualité, curriculum, delivery).
  \item \textbf{TechLead-EVM} : Référent technique EVM (Solidity, tooling, patterns, sécurité).
  \item \textbf{TechLead-SOL} : Référent technique Solana (Rust/Anchor, tooling, sécurité).
  \item \textbf{SecLead} : Référent sécurité/audit (méthodologie, checklists, hardening).
  \item \textbf{Ops} : Opérations (planning, salles, outils, LMS, comptes, supports).
  \item \textbf{Career} : Career services (coaching, portfolio, placements, partenariats RH).
  \item \textbf{Mkt} : Marketing \& admissions (funnel, contenus, événements, conversion).
  \item \textbf{Legal} : Conseil conformité (disclaimer, scénarios opératoires, risques).
  \item \textbf{Mentors} : intervenants/mentors externes (reviews, office hours, panels).
  \item \textbf{Students} : apprenants (delivery, repo, docs, demo, éthique).
\end{itemize}

\newpage
\section{Matrice SWOT (Forces, Faiblesses, Opportunités, Menaces)}

\RBKSWOT{
  \textbullet~Positionnement \textbf{premium} : ``Senior-by-Design'' (engineering, sécurité, production). \par
  \textbullet~\textbf{DualTrack EVM/Solana} : double crédibilité et employabilité internationale. \par
  \textbullet~Méthodologie \textbf{Studio} : sprints, PR reviews, CI, tests, incident drills, demo days. \par
  \textbullet~Production d'un \textbf{portfolio vérifiable} : repos, releases, docs, rubrics, capstones. \par
  \textbullet~Adossement possible à un \textbf{réseau mentors} (diaspora, builders, écosystèmes).
}{
  \textbullet~Risque de \textbf{sur-ambition} : contenu trop large si non modularisé (fatigue, dilution). \par
  \textbullet~Dépendance à des \textbf{experts rares} (Rust/Anchor, audit), risque de disponibilité. \par
  \textbullet~Exigence élevée : peut réduire le volume d’inscrits si admissions trop strictes ou discours mal cadré. \par
  \textbullet~Nécessité d’une \textbf{cohérence chiffrée} stricte (durées, prix, KPI) pour crédibilité business/investisseurs. \par
  \textbullet~Contexte local : \textbf{sensibilité réglementaire} autour des actifs numériques (communication à cadrer).
}{
  \textbullet~\textbf{Marché remote} Web3 : opportunités globales (teams distribuées, rémunérations supérieures). \par
  \textbullet~Demande croissante pour \textbf{profils sécurité / audit-readiness} (IA = plus de vulnérabilités). \par
  \textbullet~\textbf{Partenariats} (protocols, infra, wallets, analytics) : crédibilité + projets réels + recrutement. \par
  \textbullet~RBK peut devenir un \textbf{hub régional} (Afrique du Nord / francophonie) avec cohorte pilote forte. \par
  \textbullet~Monétisation additionnelle : \textbf{B2B} (formations entreprise), \textbf{studio services}, incubation.
}{
  \textbullet~Volatilité Web3 : cycles de marché \& narratives (risque sur marketing et perception). \par
  \textbullet~\textbf{Risque réputationnel} : confusion ``formation dev'' vs ``promesse crypto'' si branding imprécis. \par
  \textbullet~Risques sécurité : un capstone mal cadré peut exposer à des \textbf{mauvaises pratiques} (à prévenir). \par
  \textbullet~Concurrence MOOC/bootcamps internationaux : différenciation doit être \textbf{preuves + mentoring + studio}. \par
  \textbullet~Réglementaire : incertitudes locales (paiements, communications), besoin d’un \textbf{modèle opératoire compliant}.
}

\newpage
\section{Priorisation MoSCoW (Must, Should, Could, Won't)}

\RBKMoSCoW{
  \textbullet~Programme \textbf{DualTrack} : choix \textbf{EVM} ou \textbf{Solana} avec tronc commun + spécialisation. \par
  \textbullet~\textbf{Zéro réduction de contenu} : restructuration/ordonnancement sans suppression. \par
  \textbullet~Méthode \textbf{Studio obligatoire} : PR reviews, CI, tests minimaux, conventions repo, releases. \par
  \textbullet~\textbf{Capstones} (3) + 1 projet final : livrables, critères d’acceptation, rubric de notation. \par
  \textbullet~\textbf{Note de cadrage remplie} : SWOT/MoSCoW/RACI/registre de risques \emph{non vides} dans le corps du doc. \par
  \textbullet~\textbf{Cohérence chiffrée} : une page ``Factsheet'' unique (durée, cohorte, prérequis, prix, KPIs). \par
  \textbullet~\textbf{Conformité/Éthique} : disclaimers, anti-trading, scénarios opératoires (Tunisie / export) + risques. \par
  \textbullet~\textbf{Employabilité} : portfolio, coaching, entretiens blancs, demo day, pipeline bounties/placements.
}{
  \textbullet~Micro-certifications (badges) : critères vérifiables (repo + CI + tests + doc + demo). \par
  \textbullet~Réseau mentors : office hours mensuels + reviews structurées + panels demo day. \par
  \textbullet~Outillage standardisé : templates PR, issues, ADR, runbooks, checklists sécurité. \par
  \textbullet~Ajout de listes automatiques : \textbf{Liste des figures/tableaux} + \textbf{liste des acronymes}. \par
  \textbullet~Bibliographie / sources : section ``Sources \& hypothèses'' (marché, salaires, ROI, funnel).
}{
  \textbullet~Simulation ``incident drills'' hebdomadaire + postmortems documentés. \par
  \textbullet~Module ``Security/Audit Readiness'' avancé (option premium) : fuzzing, invariants, audit package. \par
  \textbullet~Programme ``B2B Corporate'' dérivé (formations courtes : wallet/tx, audit prep, tokenization). \par
  \textbullet~Offre ``Incubation légère'' : accompagnement 6–8 semaines post-demo day pour top projets. \par
  \textbullet~Chapitre ``Scénarios de scalabilité'' : 2/3 promos par an + ressources + QA.
}{
  \textbullet~Contenus orientés \textbf{trading/spéculation} ou promesses de gains. \par
  \textbullet~Contenus expliquant le contournement légal/réglementaire ou l’évasion fiscale. \par
  \textbullet~``Tout apprendre sur tout'' sans critères de sortie : pas de fluff, pas de slides-only. \par
  \textbullet~Déploiements mainnet non audités comme exigence pédagogique (testnet/devnet par défaut).
}

\newpage
\section{Matrice RACI (Responsable, Accountable, Consulted, Informed)}

\RBKRACI{
Définir la vision/positionnement (RBK Web3 Studio) & CEO & CEO & HoP, Mkt, Legal & Ops, Mentors, Students \\ \hline
Figer la ``Factsheet'' (durée, cohorte, prix, KPIs, tracks) & HoP & CEO & Ops, Career, Legal & Mkt, Students \\ \hline
Concevoir le tronc commun (objectifs, labs, DoD) & HoP & HoP & TechLead-EVM, TechLead-SOL, SecLead & CEO, Ops \\ \hline
Concevoir Track EVM (syllabus, labs, tooling) & TechLead-EVM & HoP & SecLead, Mentors & CEO, Ops \\ \hline
Concevoir Track Solana (syllabus, labs, tooling) & TechLead-SOL & HoP & SecLead, Mentors & CEO, Ops \\ \hline
Définir rubrics d’évaluation (PR, capstones, examens) & SecLead & HoP & TechLeads, Mentors & CEO, Students \\ \hline
Définir capstones (specs, critères d’acceptation, scoring) & HoP & HoP & TechLeads, SecLead, Career & CEO, Mkt \\ \hline
Mettre en place l’infra (LMS, repos templates, CI, comptes outils) & Ops & HoP & TechLeads & CEO, Students \\ \hline
Admissions (test d’entrée, entretiens, sélection cohorte) & Mkt & CEO & HoP, TechLeads, Career & Ops, Students \\ \hline
Encadrement hebdo (sprints, reviews, incident drills) & HoP & HoP & TechLeads, SecLead, Mentors & CEO, Ops \\ \hline
Assurance qualité (cohérence doc, ToC, versioning, build LaTeX) & Ops & HoP & SecLead & CEO, Mkt \\ \hline
Conformité \& communication (disclaimers, scénario opératoire) & Legal & CEO & HoP, Ops & Mkt, Students \\ \hline
Career services (portfolio, entretiens blancs, placements) & Career & HoP & Mentors, TechLeads & CEO, Mkt \\ \hline
Organisation Demo Day (panel, format, scoring, invitations) & Career & CEO & HoP, Mkt, Mentors & Ops, Students \\ \hline
Suivi post-cohorte (KPIs, feedback, itérations curriculum) & HoP & CEO & Ops, Career, TechLeads & Mentors, Students \\
}

\newpage
\section{Gabarits Studio (Ops \& Technique)}

Ces modèles sont utilisés quotidiennement dans le cadre des rituels "Studio".

\subsection{Template : Incident Drill Postmortem}
À remplir après chaque simulation d'attaque du Jeudi (War Room).

\begin{tcolorbox}[colback=white,colframe=red!75!black,title=Incident Postmortem \#ID]
\textbf{Date :} JJ/MM/AAAA \hfill \textbf{Severity :} Critical / High / Medium \\
\textbf{Reporter :} @StudentName \hfill \textbf{Duration :} 45min

\textbf{1. Le Scénario (What happened?)} \\
\textit{Exemple : Le smart contract permettait de retirer plus que le solde disponible (underflow).}

\textbf{2. Impact (Si Mainnet)} \\
\textit{Exemple : Perte totale des fonds du pool de liquidité (TVL = \$50k).}

\textbf{3. Root Cause Analysis (The "Why")} \\
\textit{Exemple : Utilisation de \texttt{unchecked} block dans Solidity 0.8+ sans vérification préalable.}

\textbf{4. Le Fix (Patch)} \\
\textit{Exemple : Retrait du bloc unchecked et ajout d'un \texttt{revert CustomError()}.}

\textbf{5. Lessons Learned (Prevention)} \\
\textit{Exemple : Ajouter un test Fuzzing qui tente des retraits massifs aléatoires.}
\end{tcolorbox}

\subsection{Standard Repo Structure (Template)}
Tout repository étudiant doit suivre cette structure pour être auditable.

\begin{verbatim}
/project-root
│
├── /docs               # Architecture Decision Records (ADR)
│   ├── ADR-001-Language-Choice.md
│   └── Architecture.png
│
├── /src                # Code Source
│   ├── /program        # Smart Contract (Rust/Solidity)
│   └── /client         # SDK / Frontend
│
├── /tests              # Tests d'intégration & Fuzzing
│
├── .github/workflows   # CI Scripts
│   ├── test.yml        # Unit Tests
│   └── lint.yml        # Rustfmt / Clippy
│
├── Makefile            # Commandes standard (build, test, deploy)
└── README.md           # Entry point (Install, Usage, Credits)
\end{verbatim}

\subsection{Template : Architecture Decision Record (ADR)}
Pour justifier les choix techniques structurants.

\begin{itemize}
    \item \textbf{Titre :} Choisir Anchor vs Native Rust
    \item \textbf{Statut :} Accepté
    \item \textbf{Contexte :} Nous avons besoin de développer rapidement un MVP standard.
    \item \textbf{Décision :} Utiliser Anchor Framework.
    \item \textbf{Conséquences :}
    \begin{itemize}
        \item (+) Développement plus rapide, sécurité par défaut (discriminators).
        \item (-) Abstraction magique, binaire plus lourd.
    \end{itemize}
\end{itemize}
