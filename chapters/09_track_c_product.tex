\chapter[Track C : Product Engineer]{TRACK C : PRODUCT \& GROWTH ENGINEER (FULL STACK + ANALYTICS)}
\label{chap:track_c}

\section{Philosophie du Track : Le "Product Builder" Complet}

Le monde Web3 regorge de protocoles techniquement brillants mais inutilisables. Le Track C forme le chaînon manquant : l'ingénieur capable de construire une dApp de bout en bout (Front, Indexing, Analytics) et de piloter sa croissance. Ce n'est pas un track "No-Code". C'est un track "Full-Code" orienté utilisateur. L'étudiant apprend à connecter les smart contracts au monde réel, à visualiser la data on-chain, et à itérer sur le produit en fonction des métriques.

\subsection*{Positionnement "Product Engineer"}
Il maîtrise la stack Next.js/React, l'indexation (Subgraphs/Squid), et l'analyse de données (Dune/SQL). Il sait que le code n'est qu'un moyen de livrer de la valeur.

\begin{tcolorbox}[colback=red!5,colframe=red!75!black,title=NON NÉGOCIABLE : USER-CENTRICITY]
\begin{itemize}
    \item \textbf{Zero-Friction :} Gérer les erreurs RPC, les wallets et l'UX dégradée sans perdre l'utilisateur.
    \item \textbf{Data-Driven :} Pas de feature sans tracking. Dashboard analytique jour 1.
    \item \textbf{Shipping :} Déploiement continu (Vercel/Fleek) et tests E2E (Playwright).
\end{itemize}
\end{tcolorbox}

\section{Structure Pédagogique : De l'UI à la Growth (16 Semaines)}

\begin{table}[H]
    \caption{Carte des Modules Track C}
    \centering
    \begin{tabular}{|c|l|p{4cm}|l|}
        \hline
        Sem & Module & Objectif & Livrable \\ \hline
        13-15 & Connectivity & Integration & Connect \\ \hline
    \end{tabular}
\end{table}

\subsection{MODULE 1 : Web3 Connectivity \& State Management (Semaines 13-15)}
\textbf{Objectifs :} Maîtriser la connexion Wallet (RainbowKit/Solana Adapter). Gérer l'état asynchrone (TanStack Query) et les erreurs RPC.
\textbf{Lab :} Créer un "Universal Profile Viewer" qui affiche les NFTs et soldes de n'importe quelle adresse (EVM+SVM).

\subsection{MODULE 2 : Indexing \& Data Layer (Semaines 16-18)}
\textbf{Objectifs :} La blockchain est une mauvaise base de données de lecture. Apprendre à indexer les événements Smart Contract dans une DB relationnelle (Postgres) via The Graph ou Goldsky.
\textbf{Lab :} Indexer un DEX existant (ex : Uniswap) pour requêter les volumes par paire en < 100ms.

\subsection{MODULE 3 : On-Chain Analytics (Semaines 19-21)}
\textbf{Objectifs :} SQL pour la blockchain. Utiliser Dune Analytics ou Flipside pour prouver la traction.
\textbf{Lab :} Créer un Dashboard "Whale Watcher" qui alerte sur les gros mouvements de stablecoins.

\subsection{MODULE 4 : Growth Engineering (Semaines 22-23)}
\textbf{Objectifs :} Coder la viralité. Airdrops programmatiques, Whitelists basées sur le comportement on-chain (Sybil resistance), Referral links on-chain.
\textbf{Lab :} Système de parrainage où le parrain reçoit une part des frais de transaction du filleul (Split par contrat).

\subsection{MODULE 5 : Automation \& Bots (Semaines 24-25)}
\textbf{Objectifs :} Interagir avec la blockchain sans interface web. Bots Telegram/Discord, Keepers (Chainlink Automation), Cron Jobs décentralisés.
\textbf{Lab :} Bot de sniping ou de notification de liquidation sur Aave.

\subsection{MODULE 6 : Production \& Launch (Semaines 26-28)}
\textbf{Objectifs :} Packaging final. Docs utilisateurs (GitBook), Landing page haute conversion, SEO Web3.
\textbf{Projet Final :} Un SaaS Web3 complet avec abonnement (Crypto-paiement) + Dashboard Admin.

\section{Profil de Sortie}

Un "Full Stack Web3" capable de lancer sa propre startup ou de rejoindre une équipe Growth dans un grand protocole.

\begin{table}[H]
    \caption{Matrice Compétences Product}
    \centering
    \begin{tabularx}{\textwidth}{|l|X|X|}
        \hline
        \textbf{Domaine} & \textbf{Attendu} & \textbf{Preuve} \\ \hline
        Front-end & Expert React/Next.js & App fluide 60fps \\ \hline
        Data & Capable d'écrire des subgraphs & API GraphQL \\ \hline
        Growth & Comprend les funnels & Analytics intégrés \\ \hline
    \end{tabularx}
\end{table}
