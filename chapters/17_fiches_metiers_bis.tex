\chapter{MODÈLE OPÉRATIONNEL DÉTAILLÉ}
\label{chap:finances_details}

Ce chapitre présente le détail opérationnel des investissements (CAPEX) et des charges récurrentes (OPEX) nécessaires au fonctionnement du RBK Web3 Studio.

\section{\faChartPie\ Investissement Initial (CAPEX - Phase de Lancement)}

L'investissement est réparti sur les trois mois précédant le lancement de la première promotion (Promo Alpha).

\begin{table}[ht]
\centering
\begin{tabularx}{\textwidth}{X|r|X}
\midrule
\rowcolor{SolanaPurple!20} \textbf{Poste de Dépense} & \textbf{Coût Estimé (TND)} & \textbf{Détails Techniques \& Justification} \\
\midrule
Développement du Curriculum & 15 000 & Rémunération des experts (EVM/Solana) pour la création des modules, des «~Golden Templates~» et des labs reproductibles. \\
\midrule
Équipement \& Infrastructure & 20 000 & Amélioration des postes de travail, acquisition de licences logicielles (IDE IA), et abonnement aux services de nœuds RPC premium (Helius, Alchemy, QuickNode). \\
\midrule
Marketing \& Acquisition & 15 000 & Campagnes LinkedIn/Twitter ciblées, organisation du meetup de lancement, et production de contenu de marque (vidéos, articles). \\
\midrule
\textbf{TOTAL INVESTISSEMENT} & \textbf{50 000} & Amortissement prévu sur les 3 premières promotions (16 666 TND / promo). \\
\midrule
\end{tabularx}
\caption{Investissement Initial (CAPEX)}
\end{table}

\section{\faChartPie\ Charges d'Exploitation (OPEX - Par Promotion de 20 Étudiants)}

Les charges sont calculées sur un cycle de 6-7 mois intensifs.

\subsection{A. Charges de Personnel (76 000 TND)}

\begin{table}[ht]
\centering
\begin{tabularx}{\textwidth}{X|r}
\midrule
\rowcolor{SolanaPurple!20} \textbf{Poste} & \textbf{Montant (TND)} \\
\midrule
Directeur Pédagogique / Head of Ecosystem & 40 000 \\
\midrule
Responsable Administratif \& Logistique & 24 000 \\
\midrule
Mentors Techniques (Consultants) & 12 000 \\
\midrule
\textbf{TOTAL PERSONNEL} & \textbf{76 000} \\
\midrule
\end{tabularx}
\caption{Charges de Personnel par Promotion}
\end{table}

\subsection{B. Frais de Fonctionnement et Services (35 000 TND)}

\begin{table}[ht]
\centering
\begin{tabularx}{\textwidth}{X|r}
\midrule
\rowcolor{SolanaPurple!20} \textbf{Poste} & \textbf{Montant (TND)} \\
\midrule
Services Cloud \& Outils IA & 5 000 \\
\midrule
Marketing récurrent \& Demo Day & 10 000 \\
\midrule
Frais de Certification & 10 000 \\
\midrule
Contingence (10\%) & 10 000 \\
\midrule
\textbf{TOTAL FONCTIONNEMENT} & \textbf{35 000} \\
\midrule
\end{tabularx}
\caption{Frais de Fonctionnement par Promotion}
\end{table}

\textbf{TOTAL OPEX (Variables + Fixes) :} ~111 000 TND par promotion.

\section{\faMoneyBillWave\ Modèle de Revenus et Recettes}

Le positionnement tarifaire reflète le potentiel de salaire international (5x à 10x le marché local).

\begin{table}[ht]
\centering
\begin{tabularx}{\textwidth}{X|r|r|r}
\midrule
\rowcolor{SolanaPurple!20} \textbf{Type de Billet} & \textbf{Prix Unitaire (TND)} & \textbf{Volume (Prévu)} & \textbf{Total (TND)} \\
\midrule
Prix Plein (Public) & 18 000 & 8 & 144 000 \\
\midrule
Prix Early Bird / Alumni & 15 000 & 10 & 150 000 \\
\midrule
Bourses / ISA (Pondéré) & 7 000 & 2 & 14 000 \\
\midrule
\textbf{RECETTES TOTALES (1 PROMO)} & & \textbf{20 Élèves} & \textbf{308 000} \\
\midrule
\end{tabularx}
\caption{Structure des Revenus par Promotion}
\end{table}

\section{\faChartLine\ Projection des Flux de Trésorerie (Cash Flow) - 3 Ans}

Hypothèse de croissance~: Année 1 (1 promo), Année 2 (2 promos), Année 3 (3 promos).

\begin{table}[ht]
\centering
\begin{tabularx}{\textwidth}{X|r|r|r}
\midrule
\rowcolor{SolanaPurple!20} \textbf{Indicateurs Financiers} & \textbf{Année 1 (1 Promo)} & \textbf{Année 2 (2 Promos)} & \textbf{Année 3 (3 Promos)} \\
\midrule
Nombre d'Étudiants & 20 & 50 & 75 \\
\midrule
Chiffre d'Affaires & 308 000 & 770 000 & 1 155 000 \\
\midrule
Revenus Complémentaires (Grants, Success Fees, Audit) & 10 000 & 40 000 & 100 000 \\
\midrule
\textbf{TOTAL REVENUS} & \textbf{318 000} & \textbf{810 000} & \textbf{1 255 000} \\
\midrule
OPEX (Coûts promos) & (111 000) & (250 000) & (350 000) \\
\midrule
CAPEX (Invest. initial) & (50 000) & (10 000) & (20 000) \\
\midrule
\textbf{Marge Brute (EBITDA)} & \textbf{157 000} & \textbf{550 000} & \textbf{885 000} \\
\midrule
\textbf{Bénéfice Net (Est.)} & \textbf{141 300} & \textbf{495 000} & \textbf{796 500} \\
\midrule
\end{tabularx}
\caption{Projections Financières sur 3 Ans}
\end{table}

\section{\faBalanceScale\ Analyse du Seuil de Rentabilité (Break-Even)}

\begin{itemize}
    \item Coût total pour la 1ère promo (CAPEX + OPEX)~: 161 000 TND.
    \item Revenu moyen par étudiant~: 15 400 TND.
    \item \textbf{Nombre d'étudiants pour atteindre l'équilibre (Année 1)~: 10,4 (soit 11 étudiants).}
\end{itemize}

\section{Notes et Hypothèses Stratégiques}

\subsection{Levier ISA (Income Share Agreement)}

Le modèle financier prévoit d'introduire des contrats de partage de revenus.

\begin{itemize}
    \item \textbf{Condition :} L'étudiant ne paie rien ou une partie minime, puis reverse 15\% de son salaire pendant 2 ans dès qu'il dépasse un seuil de 3 000 TND/mois.
    \item \textbf{Impact :} Augmente le volume d'étudiants d'élite et assure des revenus récurrents sur l'Année 2 et 3.
\end{itemize}

\subsection{Revenus «~Studio~» (Success Fees)}

RBK prend une participation minoritaire (5\%) dans les projets incubés qui lèvent des fonds lors du Demo Day.

\textbf{Potentiel :} Sur 75 étudiants/an, si 2 startups lèvent 100k\$ (Seed Grant), RBK perçoit 5k\$ par projet, soit ~30k TND de revenus passifs.

\subsection{Optimisation par l'IA}

Les coûts de personnel mentor sont maintenus à un niveau bas (12k par promo) car l'IA assiste 80\% des revues de code basiques, permettant aux mentors humains de se concentrer uniquement sur l'audit stratégique et l'architecture de haut niveau.

\begin{ceoBox}{Annotation pour le CEO}
Ce Business Plan montre un ROI immédiat. Le Studio s'autofinance dès le 6ème mois d'activité. La scalabilité est assurée par le passage à un modèle multi-cohortes en Année 2.
\end{ceoBox}

% ========================================
% CHAPITRE 12~: STRATÉGIE MARKETING
