\chapter{Note de cadrage — RBK 2.0 (Web3 Studio Dual-Track EVM \& Solana)}

\begin{strategieBox}{Avant-propos stratégique}
Le passage à RBK 2.0 marque une rupture avec les modèles éducatifs traditionnels. En pivotant vers une architecture de \textit{Web3 Build Studio} hybride (EVM \& Solana), nous ne formons plus de simples développeurs, mais des architectes de la souveraineté numérique. Cette note de cadrage formalise l'ambition, les moyens et le pilotage de cette transformation structurante.
\end{strategieBox}

\section{Contexte du projet}

\subsection{Situation actuelle}
RBK (ReBootKamp) s'est imposé comme un acteur clé de la formation intensive en développement logiciel. Toutefois, l'écosystème technologique mondial subit une mutation profonde avec l'avènement du Web3, de la finance décentralisée (DeFi) et des architectures distribuées. La version 1.0 du modèle a prouvé son efficacité pédagogique, mais doit désormais intégrer une dimension industrielle et commerciale directe.
L'environnement actuel se caractérise par une forte demande de profils "Seniors-by-Design" capables d'opérer sur des environnements de production critiques (Solana, Ethereum), face à une pénurie structurelle de talents vérifiés.

\subsection{Motivations et enjeux}
La motivation principale est de repositionner RBK non plus comme une école, mais comme un \textit{Talent/Venture Studio}.
L'enjeu est triple :
\begin{enumerate}
    \item \textbf{Technologique :} Maîtriser le double standard du marché (Rust/Solana pour la performance, Solidity/EVM pour l'interopérabilité).
    \item \textbf{Économique :} Sécuriser le modèle via les ISA (Income Share Agreements) et la production de valeur tangible (MVP, audits) durant la formation.
    \item \textbf{Humain :} Offrir une employabilité internationale immédiate aux talents tunisiens et africains.
\end{enumerate}

\section{Enjeux (Stratégiques, Opérationnels, Réglementaires)}

\begin{SWOTMatrix}[SWOT — Enjeux de la Transformation RBK 2.0]
\begin{itemize}[leftmargin=*]
    \item Positionnement unique "Dual-Track" en Afrique.
    \item Pédagogie immersive ("Cyborg Methodology").
    \item Modèle économique aligné (ISA).
\end{itemize}
&
\begin{itemize}[leftmargin=*]
    \item Complexité technique élevée (Rust, ZK-Proofs).
    \item Intensité du programme (risque de décrochage).
\end{itemize}
\\ \hline
\rowcolor{SolanaBlue!10}\textbf{Opportunités} & \textbf{Menaces}\\ \hline
\begin{itemize}[leftmargin=*]
    \item Demande mondiale croissante pour Solana/EVM.
    \item Partenariats institutionnels (banques, gouvernements).
    \item Vide concurrentiel sur le segment "Senior".
\end{itemize}
&
\begin{itemize}[leftmargin=*]
    \item Volatilité du marché crypto/Web3.
    \item Fuite des cerveaux sans retour sur investissement local.
    \item Évolution rapide des stacks techniques.
\end{itemize}
\\ \hline
\end{SWOTMatrix}

\section{Objectifs SMART}

Le projet RBK 2.0 s'articule autour d'objectifs précis garantissant la qualité, la maîtrise des coûts et le respect des délais.

\begin{KPITable}[Tableau de bord - Objectifs SMART]
\rowcolor{SolanaGreen!10}\textbf{Objectif} & \textbf{Définition SMART} & \textbf{Cible} & \textbf{KPI / Mesure}\\ \hline
\textbf{Excellence Tech} & Former des profils capables de déployer en mainnet & 100\% de réussite aux Capstones & Taux de déploiement Mainnet\\ \hline
\textbf{Employabilité} & Placement des talents sous 90 jours post-graduation & > 90\% & Taux d'insertion Pro\\ \hline
\textbf{Rentabilité} & ROI positif des cohortes via ISA et Factory & Break-even à M+12 & ARR / Cohorte\\ \hline
\textbf{Vélocité} & Durée de formation optimisée sans perte de qualité & 24 semaines (intense) & Time-to-Skills\\ \hline
\end{KPITable}

\subsection*{ROI et Métriques Financières}
L'hypothèse de ROI repose sur une valorisation moyenne des profils sortants à 45k€/an (marché international) et un taux de recouvrement ISA de 85\%. Le modèle "Factory" (production de MVP pour tiers) génère un revenu complémentaire estimé à 15\% du CA global.

\section{Analyse des besoins}

\subsection{Audit de l'existant}
L'infrastructure actuelle (locaux, connexion, serveurs) est robuste pour du développement Web2 classique. Le passage au Web3 nécessite une mise à niveau :
\begin{itemize}
    \item \textbf{Infrastructure Node :} Nécessité de nœuds RPC privés ou dédiés pour les tests de charge.
    \item \textbf{Sécurité :} Environnements isolés (Sandbox) pour les manipulations de smart contracts.
\end{itemize}

\subsection{Besoins Utilisateurs et Parties Prenantes}
\begin{itemize}
    \item \textbf{Apprenants (Talents) :} Recherchent une expertise rare, un mentorat de haut niveau et une insertion rapide.
    \item \textbf{Partenaires (Hiring Partners) :} Recherchent des profils "Plug \& Play", auditables via leur code sur GitHub.
    \item \textbf{Instruction Team :} A besoin d'outils de suivi automatisé (CI/CD pédagogique) et de supports à jour.
\end{itemize}

\begin{MoSCoWTable}[Priorisation des Fonctionnalités - MVP Studio]
\begin{itemize}[leftmargin=*]
\item Cursus Dual-Track complet
\item Plateforme d'évaluation auto.
\item Noeuds RPC Testnet
\end{itemize} &
\begin{itemize}[leftmargin=*]
\item Module Audit Sécurité Avancé
\item Certification Soulbound (SBT)
\end{itemize} &
\begin{itemize}[leftmargin=*]
\item Hackathons Internationaux
\item Incubateur physique dédié
\end{itemize} &
\begin{itemize}[leftmargin=*]
\item token de gouvernance DAO (V2)
\item Expansion multi-sites
\end{itemize}
\\ \hline
\end{MoSCoWTable}

\section{Solutions techniques}

\subsection{Justification du Dual-Track (EVM + Solana)}
Le choix de couvrir à la fois l'EVM (Ethereum Virtual Machine) et Solana répond à une logique de couverture de marché totale.
\begin{itemize}
    \item \textbf{EVM (Solidity/Foundry) :} Standard industriel, essentiel pour la DeFi institutionnelle et l'interopérabilité (L2s).
    \item \textbf{Solana (Rust/Anchor) :} Performance extrême, essentiel pour les applications grand public (DePIN, Payments) et l'innovation haute fréquence.
\end{itemize}

\subsection{Architecture Technique du Studio}
L'architecture du studio repose sur des principes de "DevOps-first". Chaque apprenant opère dans un conteneurisé, avec des pipelines CI/CD imposant des standards de qualité (linting, tests unitaires, couverture). L'observabilité est assurée par un dashboard centralisé suivant la progression des compétences (Skill Tree).

\section{Évaluation des risques}

La gestion des risques est intégrée dès la conception du programme ("Risk-by-Design").

\begin{RiskRegister}[Registre des Risques Prioritaires]
\rowcolor{SolanaPurple!10}\textbf{Risque} & \textbf{P} & \textbf{I} & \textbf{Mesures d'atténuation} & \textbf{Owner}\\ \hline
Saturation cognitive des apprenants & 4 & 5 & Coaching mental, Pauses actives, Suivi psy & Chief Happiness\\ \hline
Obsolescence technique rapide & 5 & 4 & Veille hebdo, Mises à jour syllabus en continu & Tech Lead\\ \hline
Défaut de paiement ISA & 3 & 4 & Sélection rigoureuse, Cadre juridique fort & Legal / Finance\\ \hline
Bugs critiques en prod (Factory) & 2 & 5 & Audits croisés, Bounties, Assurance & QA Lead\\ \hline
\end{RiskRegister}
\textit{Légende : P = Probabilité (1-5), I = Impact (1-5).}

\section{Gouvernance \& pilotage}

\subsection{Méthodologie}
Le pilotage suit une approche Agile/Scrum adaptée à la pédagogie. Des sprints de 2 semaines rythment l'apprentissage et la production.

\begin{RACITable}[Matrice RACI — Gouvernance RBK 2.0]{|Y|Z|Z|Z|Z|}
\hline
\rowcolor{SolanaBlue!10}\textbf{Activité} & \textbf{CEO} & \textbf{CTO} & \textbf{Lead Instr.} & \textbf{Ops}\\ \hline
Définition Stratégie & \raciA & \raciR & \raciC & \raciI \\ \hline
Validation Syllabus & \raciI & \raciA & \raciR & \raciI \\ \hline
Recrutement Cohorte & \raciI & \raciC & \raciC & \raciA \\ \hline
Suivi Qualité & \raciC & \raciA & \raciR & \raciI \\ \hline
Partenariats B2B & \raciA & \raciR & \raciC & \raciR \\ \hline
\end{RACITable}

\section{Planification, budget, indicateurs}

\subsection{Rétroplanning des Grands Jalons}
\begin{itemize}
    \item \textbf{Mois 1-2 :} Finalisation Ingénierie Pédagogique \& Recrutement Staff.
    \item \textbf{Mois 3 :} Lancement Campagne Candidats (Marketing).
    \item \textbf{Mois 4 :} Sélection \& Bootcamps pré-rentrée.
    \item \textbf{Mois 5 :} KICK-OFF Cohorte #1 (S0).
    \item \textbf{Mois 11 :} Demo Day \& Graduation.
\end{itemize}

\subsection{Budget et Coûts}
Le budget prévisionnel distingue les CAPEX (Infrastructure matériel, Contenu propriétaire) des OPEX (Salaires staff, Marketing, Cloud). Une provision pour risque de 10\% est intégrée.

\section{Conduite du changement}

La transformation vers RBK 2.0 demande un accompagnement soutenu :
\begin{itemize}
    \item \textbf{Formation des formateurs :} Montée en compétence obligatoire sur Rust et Solidity Avancé.
    \item \textbf{Communication :} Clarifier le passage d'une "école de code" à un "Centre d'Excellence Web3".
    \item \textbf{Adhésion :} Impliquer les anciens (Alumni) comme mentors pour faciliter la transition culturelle.
\end{itemize}

\section{Conclusion de la Note}

Cette note de cadrage valide la faisabilité et la pertinence du pivot RBK 2.0. En alignant l'excellence technique sur la réalité du marché Web3, RBK se dote d'un avantage concurrentiel décisif. La structure Dual-Track, soutenue par une gouvernance rigoureuse et une gestion des risques proactive, assure la pérennité du modèle.

\begin{roadmapBox}{Recommandation}
Il est recommandé de VALIDER ce cadrage et de lancer immédiatement la phase d'exécution (Recrutement Staff Technique & Préparation Infrastructure).
\end{roadmapBox}
