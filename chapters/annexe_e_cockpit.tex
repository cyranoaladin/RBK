\chapter{LE COCKPIT DE L'ARCHITECTE}

Liste des outils obligatoires pour un étudiant en phase de production.

\section{Stack Outillage Minimal}

\begin{center}
    \textbf{Table: Cockpit Tools} \\
    \small
    \begin{tabular}{|l|l|p{6cm}|}
        \hline
        \textbf{Outil} & \textbf{Usage} & \textbf{Output Attendu} \\ \hline
        \textbf{Obsidian/Notion} & Knowledge Base & Wiki du projet, Notes de recherche \\ \hline
        \textbf{Excalidraw} & Diagramming & Schémas d'architecture C4 \\ \hline
        \textbf{Linear/Jira} & Task Management & Tickets spécifiés et trackés \\ \hline
        \textbf{Cursor/VSCode} & IDE & Code avec Linter et Copilot configuré \\ \hline
    \end{tabular}
\end{center}

\section{Journée Type (Productivité)}

\begin{itemize}
    \item \textbf{09h-12h (Deep Work) :} Coding (Feature complexe ou Refactoring). Pas de notifs.
    \item \textbf{13h-14h (Review) :} Code Review des PRs des collègues.
    \item \textbf{14h-16h (Ops) :} Tests, Documentation, Fixes mineurs.
    \item \textbf{16h-17h (Sync) :} Daily Standup, Synchro Architecte.
\end{itemize}
