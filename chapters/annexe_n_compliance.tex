\chapter{GUIDE DE CONFORMITÉ WEB3}

Ce guide est distribué à tous les diplômés pour assurer leur sécurité juridique lors de leurs premières missions freelances internationales.

\section{Statuts Juridiques en Tunisie}

\begin{itemize}
    \item \textbf{Personne Physique (Patente) :} Simple, régime réel ou forfaitaire. Idéal pour débuter. Code activité : "Exportation de Services Informatiques".
    \item \textbf{SUARL (Société Unipersonnelle) :} Plus lourdeur administrative, mais responsabilité limitée et crédibilité B2B accrue.
\end{itemize}

\section{Gestion des Crypto-Actifs}

\textit{Avertissement : Ce contenu ne constitue pas un conseil juridique. Consulter un expert comptable.}

\begin{enumerate}
    \item \textbf{Facturation :} Toujours facturer en Devises (EUR/USD) ou équivalent stablecoin, avec une trace contractuelle (Contrat de prestation).
    \item \textbf{Rpatriement :} La loi oblige le rapatriement des revenus dexportation. Lusage dintermédiaires agréés (ex: Deel, Rise, ou comptes bancaires devises) est recommandé pour convertir Crypto $\to$ Fiat avant larrivée en Tunisie.
    \item \textbf{HODL :} Conserver une partie de sa rémunération en crypto pour spéculer est une activité de gestion de patrimoine privée, distincte de lactivité pro.
\end{enumerate}
