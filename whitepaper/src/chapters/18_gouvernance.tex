\chapter{GOUVERNANCE, ÉTHIQUE \& TRANSPARENCE}

RBK 2.0 aspire à devenir une institution de confiance. Cela exige une gouvernance partagée et une transparence radicale sur nos résultats.

\section{Comité Éthique \& Pédagogique (CEP)}

Le CEP est l'organe de contre-pouvoir indépendant qui garantit que l'école reste fidèle à sa mission.

\subsection{Composition (5 Membres)}
\begin{enumerate}
    \item \textbf{Président :} Une figure de la Tech en Tunisie (ex: CTO d'une Startup à succès, non lié à RBK).
    \item \textbf{Représentant Alumni :} Élu par la DAO des anciens.
    \item \textbf{Représentant Étudiants :} Délégué de la promo en cours.
    \item \textbf{Expert Éducation :} Un pédagogue ou universitaire.
    \item \textbf{Observateur Foundation :} Un membre de la Solana Foundation (Rôle consultatif).
    \item \textbf{CEO RBK :} Voix consultative (ne vote pas sur sa propre rémunération ou sa révocation).
\end{enumerate}

\subsection{Mandat}
Le CEP se réunit trimestriellement pour :
\begin{itemize}
    \item Valider les changements majeurs de curriculum.
    \item Arbitrer les contentieux ISA complexes (ex: demande de grâce pour cas de force majeure).
    \item Auditer les taux de placement déclarés.
\end{itemize}

\section{Transparence Radicale (Open Metrics)}

Contrairement aux écoles opaques, RBK publie ses KPI en temps réel sur une page publique "Status" (et on-chain).

\begin{techBox}{Metriques Publiques (Dashboard)}
    \begin{itemize}
        \item \textbf{Taux de Placement Réel :} Calculé à J+180 (CDI/Freelance).
        \item \textbf{Salaire Médian de Sortie :} Basé sur les fiches de paie anonymisées.
        \item \textbf{Taux de Remboursement ISA :} \% de recouvrement (indicateur de santé financière).
        \item \textbf{Diversité :} Ratio Homme/Femme et Répartition Géographique (Hors Grand Tunis).
    \end{itemize}
\end{techBox}

\section{Charte de Déontologie}

RBK s'engage formellement sur les points suivants :

\begin{enumerate}
    \item \textbf{Pas de Diplôme de Complaisance :} Un étudiant qui paie Upfront mais échoue aux examens techniques ne reçoit PAS de certification. Le niveau ne s'achète pas.
    \item \textbf{Consentement Éclairé ISA :} Chaque candidat reçoit une simulation "Worst Case" (Salaire élevé = Paiement max) avant de signer.
    \item \textbf{Neutralité Technologique :} Bien que financés par des écosystèmes (ex: Solana), nous enseignons l'ingénierie fondamentale, pas le dogmatisme. Nous critiquons les faiblesses de chaque chaine.
    \item \textbf{Protection des Données :} Refus de monétiser les données étudiants auprès de recruteurs tiers sans "Opt-in" spécifique.
\end{enumerate}

\section{Structure Juridique et Rôles (Branding)}

Pour assurer une clarté totale vis-à-vis des étudiants et partenaires, nous distinguons les entités comme suit :

\begin{description}
    \item[ReBootKamp (RBK) Tunisie :] L'entité légale opératrice historique. Elle porte l'agrément de formation, gère les locaux (Ariana), les contrats étudiants (ISA inclus via véhicule dédié) et le staff administratif. C'est le garant de la conformité locale.
    
    \item[Money Factory AI :] Le partenaire technologique et pédagogique exclusif. Basé à Dubai/Singapour, Money Factory fournit le curriculum "Cyborg", la plateforme LMS propriétaire, l'infrastructure de certification On-Chain (SBT) et l'accès au réseau international (Superteam, VCs).
    
    \item[Le Programme "RBK 2.0" :] Est le fruit de cette joint-venture : l'infrastructure opérationnelle de RBK propulsée par l'expertise technique de Money Factory AI.
\end{description}
