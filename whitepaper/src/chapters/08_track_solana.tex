\chapter{TRACK A : SOLANA SMART CONTRACT ENGINEER (RUST/ANCHOR)}

% =========================================================
\section{Résumé exécutif du track (1 page)}
\label{sec:exec_summary}

\paragraph{Positionnement.}
Ce track forme un \textbf{Smart Contract Engineer Solana} capable de livrer des programmes \textbf{audit-ready}, conscients des contraintes \textbf{Account Model / CPI / compute units} et de la réalité \textbf{production} (observabilité, incidents, UX transactionnelle, reproductibilité). Le profil de sortie visé est le \textbf{Guardian} : un ingénieur qui sait \emph{concevoir, implémenter, tester, documenter, auditer, durcir et opérer} un smart contract Solana.

\paragraph{Promesse mesurable.}
À la fin du track, l'apprenant est capable de :
\begin{itemize}
  \item produire un \textbf{repo studio-grade} avec \textbf{tests automatisés}, \textbf{documentation}, \textbf{scripts reproductibles}, \textbf{CI} et \textbf{runbook},
  \item présenter un \textbf{threat model} (STRIDE simplifié) et un \textbf{mini-audit report} (findings classés, correctifs, preuves),
  \item maîtriser les patterns \textbf{PDA / seeds / constraints / CPI} et éviter les anti-patterns de sécurité,
  \item livrer un \textbf{capstone de production} incluant observabilité, critères d'acceptation et performance budget (compute).
\end{itemize}

\paragraph{Pré-requis.}
\begin{itemize}
  \item Rust niveau intermédiaire (ownership/borrowing, Result, modules, tests).
  \item Bases blockchain : transactions, signatures, état, events/logs.
  \item Discipline d'ingénierie : Git, PRs, revue, documentation.
\end{itemize}

\paragraph{Différenciation pédagogique (non négociable).}
\begin{itemize}
  \item \textbf{Spec-first} : aucun lab sans spécification, critères d'acceptation et plan de tests.
  \item \textbf{Security-first} : threat model et checklists à chaque jalon.
  \item \textbf{Reproductibilité} : scripts de build/test, environnements et démos rejouables.
  \item \textbf{Production mindset} : observabilité, runbook, erreurs explicites, UX transaction.
\end{itemize}

\begin{tcolorbox}[title={Ce que ce track n'est pas}]
Ce track \textbf{n'est pas} un apprentissage ``rapide'' centré sur la syntaxe. Il vise la capacité à \textbf{livrer} des programmes Solana robustes, testés, documentés et \textbf{auditable} --- c'est-à-dire utilisables dans un contexte professionnel.
\end{tcolorbox}

% =========================================================
\section{Objectifs mesurables et preuves attendues}
\label{sec:objectifs}

\paragraph{Règle.} Chaque compétence est évaluée par une \textbf{preuve vérifiable} (artefact), un \textbf{seuil minimal} et un \textbf{outil de vérification}.

\begin{table}[h]
\centering
\begin{tabularx}{\textwidth}{lY Y Y}
\toprule
\textbf{Domaine} & \textbf{Compétence} & \textbf{Preuve attendue} & \textbf{Outil \& seuil minimal} \\
\midrule
Solana Model & Account model correct (owners, signers, rent/lamports, sérialisation) &
Schéma des comptes + tests de cas limites + logs explicites &
Anchor/solana test + couverture tests \(\geq\) 70\% des branches critiques \\
Anchor & PDAs/constraints robustes (seeds, bumps, ownership checks) &
IDL + tests négatifs + checklist seeds/constraints signée &
Anchor tests + revue ``security gate'' validée \\
CPI & Composabilité via CPI sans fuite d'autorité &
Schéma call-graph + tests CPI + restrictions d'authority &
Tests d'intégration + audit checklist ``CPI safety'' \\
Sécurité & Threat model (STRIDE) + findings &
Document threat model + mini audit report (min 6 findings dont 2 ``High'') &
Template audit + preuves (PoC tests) \\
Qualité & CI/CD + reproductibilité &
Pipeline CI + scripts build/test + README exécutable &
CI verte + ``fresh clone works'' (checklist) \\
Production & Observabilité + runbook incidents &
Dashboard (métriques/alertes) + runbook + scénarios incidents simulés &
PRR validée + MTTR cible simulée \(\leq\) 30 min \\
Performance & Compute-aware + optimisation raisonnable &
Benchmarks CU + justification + budget perf &
Rapport perf + respect budget CU défini \\
\bottomrule
\end{tabularx}
\caption{Compétences cibles vs preuves vérifiables}
\label{tab:competences_preuves}
\end{table}

% =========================================================
\section{Programme (12 semaines : Semaines 9--20) --- modules 1 à 4}
\label{sec:programme}

\paragraph{Principe.} Chaque semaine produit un \textbf{artefact portfolio}. Chaque module se termine par un \textbf{jalon} avec \textbf{criteria gate} (Go/No-Go).

\begin{table}[H]
\centering
\caption{Semaine $\rightarrow$ objectifs $\rightarrow$ lab $\rightarrow$ livrable $\rightarrow$ DoD}
\label{tab:programme_12semaines}
\small
\begin{tabularx}{\textwidth}{l p{2.8cm} X p{3.2cm} p{2.8cm}}
\toprule
\textbf{Sem} & \textbf{Focus} & \textbf{Objectifs} & \textbf{Lab / livrable} & \textbf{DoD} \\
\midrule
S9 & Solana modèle & Transac, instr, comptes, signers & Lab A v0 : Board & Unit tests \\
S10 & Sécurité base & Owner, signers, logs & Lab A v1 : Perms & Checklist \\
S11 & Vault/Escrow & Dépôts, release, cas limites & Lab B : Escrow & Spec + Tests e2e \\
S12 & Anchor basics & Macros, constraints, IDL & Lab C : Counter & IDL + Tests \\
S13 & PDAs & Seeds, auth, constraints & Lab D : Staking & Tests négatifs \\
S14 & Composabilité & CPI, events, patterns & Livrable M2 & Gate M2 \\
S15 & Archi avancée & CPI orchestrator & Lab E : CPI & Call graph \\
S16 & Token 2022 & Ext, metadata, policies & Lab F : Policy & ADR + Tests \\
S17 & Innovation & Indexing, UX, Blinks & Livrable M3 & Gate M3 \\
S18 & Hardening & Runbook, metrics & Capstone v0 & Observabilité \\
S19 & Perf/Sécu & Compute budget, load test & Capstone v1 & Budget CU \\
S20 & Release & CI, docs, demo & Capstone final & Gate final \\
\bottomrule
\end{tabularx}
\end{table}

% =========================================================
\section{Labs détaillés et critères d'acceptation}
\label{sec:labs}

\subsection*{Règles communes à tous les labs}
\begin{itemize}
  \item Chaque lab commence par une \textbf{spec} (fonctionnalités, comptes, invariants, erreurs).
  \item Les tests incluent \textbf{cas nominaux} + \textbf{cas négatifs} (attaques triviales, permissions, double-spend, etc.).
  \item Chaque livrable inclut : \textbf{README exécutable}, \textbf{scripts}, \textbf{schéma des comptes}, \textbf{journal des décisions} (ADR).
\end{itemize}

% ---------------------------
\subsection*{Lab A --- Message Board (natif Solana)}
\paragraph{Objectif.} Construire une primitive on-chain simple, mais \textbf{correcte} : stockage sérialisé, authorisation, erreurs explicites, logs.

\begin{table}[h]
\centering
\begin{tabularx}{\textwidth}{Y Y Y}
\toprule
\textbf{Fonctionnalité} & \textbf{Entrées / Comptes} & \textbf{Tests attendus (mesurables)} \\
\midrule
Créer un message & signer auteur + compte message (PDA ou compte dédié) &
création OK ; échec si auteur absent ; taille max contrôlée \\
Modifier un message & signer auteur + compte message &
échec si non-auteur ; logs explicites ; état cohérent \\
Lister via indexation off-chain & events/logs + indexer (simulé) &
events émis ; schéma d'event stable ; doc d'intégration \\
Gestion erreurs & codes d'erreur + messages &
tests sur erreurs attendues ; pas d'erreurs silencieuses \\
\bottomrule
\end{tabularx}
\caption{Spécification Lab A (Message Board)}
\label{tab:labA_spec}
\end{table}

\paragraph{Critères d'acceptation (Gate Lab A).}
\begin{itemize}
  \item README ``fresh clone'' : un clone vierge compile et passe les tests.
  \item Tests : \(\geq\) 12 tests dont \(\geq\) 5 négatifs.
  \item Documentation : schéma des comptes + table ``Instruction $\rightarrow$ comptes $\rightarrow$ erreurs''.
\end{itemize}

% ---------------------------
\subsection*{Lab B --- Mini Escrow / Vault (natif Solana)}
\paragraph{Objectif.} Implémenter un escrow/vault avec règles de release et \textbf{anti-abus}.

\begin{table}[h]
\centering
\begin{tabularx}{\textwidth}{Y Y Y}
\toprule
\textbf{Règle} & \textbf{Invariants} & \textbf{Tests attendus} \\
\midrule
Dépôt & fonds conservés dans vault ; owner vérifié &
dépôt OK ; échec si mauvais owner ; double dépôt contrôlé \\
Release conditionnel & release uniquement si condition satisfaite &
release OK ; échec si condition fausse ; reentrancy-like patterns (simulés) \\
Cancel / timeout & cancel seulement par rôle autorisé / selon délai &
cancel OK ; échec si rôle incorrect ; délais testés \\
Journalisation & events pour opérations critiques &
events présents ; payload stable ; doc d'event \\
\bottomrule
\end{tabularx}
\caption{Invariants et tests Lab B (Escrow/Vault)}
\label{tab:labB_invariants}
\end{table}

% ---------------------------
\subsection*{Lab C --- Counter+Vault (Anchor fundamentals)}
\paragraph{Objectif.} Maîtriser IDL, macros, constraints, events, error codes.

\begin{table}[h]
\centering
\begin{tabularx}{\textwidth}{Y Y Y}
\toprule
\textbf{Instruction} & \textbf{Comptes (résumé)} & \textbf{Erreurs \& tests} \\
\midrule
initialize & authority signer + state PDA &
échec si déjà init ; seeds correctes \\
increment & authority signer + state PDA &
échec si non-authority ; event émis \\
deposit/withdraw & user signer + vault + state &
échec si fonds insuffisants ; logs ; invariants \\
\bottomrule
\end{tabularx}
\caption{IDL et surface API Lab C}
\label{tab:labC_idl}
\end{table}

% ---------------------------
\subsection*{Lab D --- Staking minimal (Anchor PDAs \& constraints)}
\paragraph{Objectif.} PDAs déterministes + constraints robustes + tests négatifs.

\paragraph{Critères d'acceptation.}
\begin{itemize}
  \item Tous les accès sensibles protégé par checks explicites (authority/owner/signers).
  \item Seeds documentées ; tests sur seeds ; ``bump mismatch'' géré.
  \item Tests de retrait anticipé/illégitime \(\rightarrow\) échec explicite.
\end{itemize}

% ---------------------------
\subsection*{Lab E --- CPI composability challenge (Architecture avancée)}
\paragraph{Objectif.} Construire un orchestrateur qui appelle un programme ``service'' via CPI en conservant un modèle d'autorité sain.

\begin{tcolorbox}[title={Exigences CPI (non négociables)}]
\begin{itemize}
  \item Aucun CPI ne doit introduire une autorité implicite ou réutilisable par un tiers.
  \item Les comptes en CPI doivent être minimaux et vérifiés (owner, seeds, signers).
  \item Le call-graph est documenté et testé.
\end{itemize}
\end{tcolorbox}

% ---------------------------
\subsection*{Lab F --- Token extension / policy (Token-2022 ou équivalent)}
\paragraph{Objectif.} Démontrer la capacité à gérer des métadonnées/policies et leur validation (off-chain + on-chain checks).

\paragraph{Livrables.}
\begin{itemize}
  \item ADR : choix d'extension/pattern.
  \item Tests : policy enforcement (positifs + négatifs).
  \item Guide intégration client (front) : ``how to verify''.
\end{itemize}

% =========================================================
\section{Rubrique de notation (standard audit) --- total 100}
\label{sec:rubrique}

\begin{table}[h]
\centering
\begin{tabularx}{\textwidth}{l c Y}
\toprule
\textbf{Axe} & \textbf{Poids} & \textbf{Critères (exemples mesurables)} \\
\midrule
Sécurité \& threat model & 25 &
threat model complet ; findings classés ; corrections justifiées ; tests ``négatifs'' \\
Tests \& qualité & 20 &
tests unit+int ; cas limites ; CI stable ; couverture sur chemins critiques \\
Architecture \& invariants & 15 &
invariants explicités ; schémas comptes ; call-graph CPI ; décisions ADR \\
Production \& observabilité & 15 &
runbook ; logs ; métriques ; alerting ; scénarios incidents simulés \\
Performance (compute) & 10 &
budget CU ; benchmarks ; choix d'optimisation argumentés \\
Docs \& demo & 15 &
README exécutable ; doc API ; diagrammes ; démo reproductible \\
\bottomrule
\end{tabularx}
\caption{Rubrique d'évaluation Track A (100 points)}
\label{tab:rubric_trackA}
\end{table}

\begin{tcolorbox}[title={Condition d'échec immédiate (Fail conditions)}]
\begin{itemize}
  \item absence de tests négatifs sur les chemins d'autorité,
  \item absence de documentation des seeds/PDA sur un programme Anchor,
  \item livrable non reproductible (fresh clone ne compile pas / tests KO),
  \item absence de threat model sur capstone.
\end{itemize}
\end{tcolorbox}

% =========================================================
\section{Stack outillage et standards repo}
\label{sec:stack}

\begin{table}[h]
\centering
\begin{tabularx}{\textwidth}{l Y Y}
\toprule
\textbf{Catégorie} & \textbf{Outils} & \textbf{Standard attendu (artefact)} \\
\midrule
Toolchain & Rust toolchain, Solana CLI, Anchor &
scripts build/test ; versions documentées ; ``fresh clone'' \\
Testing & anchor test, tests TS, tests Rust &
unit + integration + négatifs ; rapports CI \\
Qualité & fmt/clippy, lint, pre-commit &
format stable ; lint clean ; conventions repo \\
Observabilité & logs, métriques, dashboards (selon stack) &
runbook + alert rules + tableau ``signaux'' \\
Docs & README, schémas, ADR, threat model &
dossier /docs ; templates remplis ; lien demo \\
\bottomrule
\end{tabularx}
\caption{Stack Track A --- outillage et standards}
\label{tab:stack_trackA}
\end{table}

\begin{tcolorbox}[title={Standards repo (minimum)}]
\begin{itemize}
  \item \texttt{/programs} (code on-chain), \texttt{/tests}, \texttt{/scripts}, \texttt{/docs}
  \item \texttt{README.md} exécutable + prérequis + ``how to run tests''
  \item \texttt{docs/architecture.md}, \texttt{docs/threat-model.md}, \texttt{docs/adr/}
  \item CI : lint + tests + artefacts (rapport) ; badge de status
\end{itemize}
\end{tcolorbox}

% =========================================================
\section{Portfolio minimal et employability pack}
\label{sec:portfolio}

\paragraph{Portfolio minimal (obligatoire).}
\begin{itemize}
  \item \textbf{Repo 1 --- Solana Native Primitive} : Lab A/B finalisé, tests négatifs, doc comptes.
  \item \textbf{Repo 2 --- Anchor Program} : IDL propre, PDAs/constraints robustes, guide intégration client.
  \item \textbf{Repo 3 --- Capstone Studio} : hardening + observabilité + perf budget + audit report.
  \item \textbf{1 mini audit report} : findings, sévérité, correctifs, preuves tests.
  \item \textbf{1 runbook} : incidents types + actions + métriques/alerting.
\end{itemize}

\paragraph{Employability pack (recruteur-ready).}
\begin{itemize}
  \item 1 page ``\emph{What I shipped}'' (liens repos + démos).
  \item 1 page ``\emph{Security posture}'' (threat model + checklists + exemples corrections).
  \item 1 page ``\emph{Performance posture}'' (compute budgets, benches, arbitrages).
\end{itemize}

% =========================================================
\section{Annexes du track (templates \& checklists)}
\label{sec:annexes_trackA}

\subsection*{A) Security checklist Solana/Anchor (réutilisable)}
\begin{table}[h]
\centering
\begin{tabularx}{\textwidth}{Y Y}
\toprule
\textbf{Contrôle} & \textbf{Comment vérifier (preuve)} \\
\midrule
Owner checks & asserts explicites + tests négatifs \\
Signer checks & requires signer + tests ``missing signer'' \\
PDA seeds/bump & seeds documentées + tests seeds + collisions évitées \\
Account constraints (Anchor) & constraints justifiées + tests violations \\
CPI safety & call-graph + accounts minimal + authority non escaladable \\
Error taxonomy & erreurs explicites + mapping doc + logs \\
Replay / double spend logique & invariants + tests de répétition \\
Overflow / underflow (logique) & checks + tests limites \\
State machine cohérente & diagramme + tests transitions invalides \\
\bottomrule
\end{tabularx}
\caption{Security checklist Solana/Anchor}
\label{tab:security_checklist_trackA}
\end{table}

\subsection*{B) Template ADR (Architecture Decision Record)}
\begin{tcolorbox}[title={ADR Template (copier-coller)}]
\textbf{Titre :} ADR-\#\# --- \emph{[Décision]}\\
\textbf{Contexte :} \emph{[Problème, contraintes]}\\
\textbf{Options :} \emph{[Option A / B / C]}\\
\textbf{Décision :} \emph{[Choix]}\\
\textbf{Rationale :} \emph{[Pourquoi]}\\
\textbf{Conséquences :} \emph{[Trade-offs]}\\
\textbf{Risques :} \emph{[Risques]}\\
\textbf{Mitigations :} \emph{[Contrôles/tests]}\\
\end{tcolorbox}

\subsection*{C) Template Threat Model (STRIDE simplifié)}
\begin{tcolorbox}[title={Threat Model Template (STRIDE simplifié)}]
\textbf{Système :} \emph{[Nom + périmètre]}\\
\textbf{Actifs critiques :} \emph{[fonds, autorisations, état]}\\
\textbf{Hypothèses :} \emph{[RPC, indexer, user behavior]}\\
\textbf{Surfaces d'attaque :} \emph{[instructions, CPI, accounts]}\\
\textbf{Menaces (STRIDE) :}\\
- Spoofing : \emph{[...]}\\
- Tampering : \emph{[...]}\\
- Repudiation : \emph{[...]}\\
- Information disclosure : \emph{[...]}\\
- Denial of service : \emph{[...]}\\
- Elevation of privilege : \emph{[...]}\\
\textbf{Contrôles :} \emph{[checks, constraints, tests]}\\
\textbf{Tests associés :} \emph{[négatifs, invariants]}\\
\end{tcolorbox}

\subsection*{D) PRR --- Production Readiness Review (tableau)}
\begin{table}[h]
\centering
\begin{tabularx}{\textwidth}{l Y Y}
\toprule
\textbf{Domaine} & \textbf{Critère} & \textbf{Preuve} \\
\midrule
Qualité & CI verte + tests + lint clean & logs CI + badge + rapports \\
Sécurité & threat model + checklist signée & docs + PR review \\
Docs & README exécutable + API surface & /docs + guide \\
Observabilité & logs + métriques + alertes & dashboard + runbook \\
Perf & budget CU + bench & rapport bench \\
Release & versioning + changelog & tag + notes \\
\bottomrule
\end{tabularx}
\caption{PRR (Production Readiness Review) --- Track A}
\label{tab:prr_trackA}
\end{table}

% =========================================================
\section{Figures indispensables (TikZ)}
\label{sec:figures}

\subsection*{Figure 1 --- Account model / instruction flow}
\begin{figure}[h]
\centering
\begin{tikzpicture}[
  node distance=10mm,
  box/.style={draw, rounded corners, align=center, minimum width=34mm, minimum height=10mm},
  arr/.style={-Latex, thick}
]
\node[box] (user) {Utilisateur\\(wallet)};
\node[box, right=18mm of user] (tx) {Transaction\\+ signatures};
\node[box, right=18mm of tx] (program) {Program\\Solana};
\node[box, below=10mm of program] (accounts) {Accounts\\(state, vault, PDA)};
\node[box, below=10mm of tx] (rpc) {RPC / Runtime};

\draw[arr] (user) -- (tx);
\draw[arr] (tx) -- (rpc);
\draw[arr] (rpc) -- (program);
\draw[arr] (program) -- (accounts);
\draw[arr] (accounts) -- (program);
\end{tikzpicture}
\caption{Account model / instruction flow (vue simplifiée)}
\label{fig:account_model_flow}
\end{figure}

\subsection*{Figure 2 --- CPI call graph (simplifié)}
\begin{figure}[h]
\centering
\begin{tikzpicture}[
  node distance=12mm,
  box/.style={draw, rounded corners, align=center, minimum width=34mm, minimum height=10mm},
  arr/.style={-Latex, thick}
]
\node[box] (A) {Programme A\\(Orchestrator)};
\node[box, right=20mm of A] (B) {Programme B\\(Service)};
\node[box, right=20mm of B] (T) {Token Program\\/ Sys Program};

\draw[arr] (A) -- node[above]{CPI} (B);
\draw[arr] (B) -- node[above]{CPI} (T);

\node[below=8mm of A, align=left] (note) {\small \textbf{Règle :} comptes minimaux, owner checks,\\\small authority explicitement contrôlée.};
\end{tikzpicture}
\caption{CPI call graph (orchestrateur $\rightarrow$ service $\rightarrow$ programme système)}
\label{fig:cpi_call_graph}
\end{figure}
