\chapter{GLOSSAIRE COMPLET}

\section{Concepts Fondamentaux Web3}

\begin{center}
    \small
    \rowcolors{2}{gray!5}{white}
    \renewcommand{\arraystretch}{1.4}
    \begin{tabularx}{\textwidth}{|l|X|}
        \hline
        \textbf{Terme} & \textbf{Définition} \\ \hline
        \textbf{Web3} & La 3ème itération d'Internet, décentralisée et basée sur la propriété numérique via la blockchain (vs Web2 dominé par les plateformes centralisées). \\ \hline
        \textbf{Blockchain} & Un registre numérique partagé, immuable et distribué qui enregistre les transactions et suit les actifs d'un réseau. \\ \hline
        \textbf{Smart Contract} & Programme informatique auto-exécutable stocké sur une blockchain qui s'exécute lorsque des conditions prédéfinies sont remplies. \\ \hline
        \textbf{DApp} & Application Décentralisée fonctionnant sur une blockchain via des Smart Contracts, sans serveur central de contrôle. \\ \hline
        \textbf{Tokenomics} & L'économie d'un token : son émission, sa distribution, son utilité et les mécanismes d'incitation financière. \\ \hline
        \textbf{DAO} & Organisation Autonome Décentralisée : Une entité gérée par du code (Smart Contracts) et gouvernée par ses membres via des tokens. \\ \hline
    \end{tabularx}
\end{center}

\section{Infrastructure \& Protocoles}

\begin{center}
    \small
    \rowcolors{2}{gray!5}{white}
    \renewcommand{\arraystretch}{1.4}
    \begin{tabularx}{\textwidth}{|l|X|}
        \hline
        \textbf{Terme} & \textbf{Définition} \\ \hline
        \textbf{Layer 1 (L1)} & Blockchain principale (ex: Solana, Ethereum) qui assure la sécurité et le consensus. \\ \hline
        \textbf{Layer 2 (L2) / Rollup} & Solution de mise à l'échelle construite "par-dessus" un L1 (ex: Ethereum) pour réduire les coûts et augmenter la vitesse. \\ \hline
        \textbf{EVM} & Ethereum Virtual Machine : L'environnement d'exécution standard d'Ethereum, utilisé aussi par de nombreuses autres chaînes (Polygon, Base). \\ \hline
        \textbf{SVM} & Solana Virtual Machine : Moteur d'exécution haute performance de Solana, capable de traiter des milliers de transactions en parallèle. \\ \hline
        \textbf{DePIN} & Decentralized Physical Infrastructure Networks : Utilisation de la blockchain pour gérer des infrastructures physiques (télécoms, énergie, GPU). \\ \hline
        \textbf{DeFi} & Finance Décentralisée : Services financiers (prêt, échange) sans intermédiaires bancaires. \\ \hline
        \textbf{Oracle} & Service tiers qui connecte les Smart Contracts aux données du monde réel (prix, météo). \\ \hline
        \textbf{Bridge} & Protocole permettant de transférer des actifs ou des données entre deux blockchains différentes. \\ \hline
    \end{tabularx}
\end{center}

\section{Terminologie Solana (Spécifique)}

\begin{center}
    \small
    \rowcolors{2}{gray!5}{white}
    \renewcommand{\arraystretch}{1.4}
    \begin{tabularx}{\textwidth}{|l|X|}
        \hline
        \textbf{Terme} & \textbf{Définition} \\ \hline
        \textbf{Account Model} & Modèle de données où tout est un "Compte" (Fichiers, Programmes, Données). Contraire au modèle UTXO de Bitcoin. \\ \hline
        \textbf{PDA} & Program Derived Address : Une adresse contrôlée par un programme (non par une clé privée), essentielle pour la sécurité et l'automatisation. \\ \hline
        \textbf{CPI} & Cross-Program Invocation : Capacité d'un programme à en appeler un autre (composabilité). \\ \hline
        \textbf{Sealevel} & Le moteur de parallélisation de Solana qui permet d’exécuter des smart contracts simultanément. \\ \hline
        \textbf{SBT} & Soulbound Token : Token non-transférable lié à l'identité (numérique) d'une personne, utilisé pour les certificats/diplômes. \\ \hline
    \end{tabularx}
\end{center}

\section{Business \& Métier}

\begin{center}
    \small
    \rowcolors{2}{gray!5}{white}
    \renewcommand{\arraystretch}{1.4}
    \begin{tabularx}{\textwidth}{|l|X|}
        \hline
        \textbf{Terme} & \textbf{Définition} \\ \hline
        \textbf{ISA} & Income Share Agreement : Accord de partage de revenus où l'étudiant paie sa formation après l'embauche. \\ \hline
        \textbf{Gas} & Frais payés au réseau pour exécuter une transaction ou un contrat. \\ \hline
        \textbf{Audit} & Examen de sécurité approfondi du code d'un Smart Contract par des experts tiers. \\ \hline
        \textbf{TVL} & Total Value Locked : Valeur totale des actifs déposés dans un protocole DeFi (indicateur de succès). \\ \hline
    \end{tabularx}
\end{center}
