\chapter{SYLLABUS TECHNIQUE COMPLET (28 SEMAINES)}

\textit{Note : Ce chapitre détaille l'exécution technique. L'intégration des Soft Skills (S25-S28) est traitée au Chapitre 7.}

\section{Calendrier Pédagogique Global}

\begin{figure}[h]
    \centering
    \begin{tikzpicture}[xscale=0.55, every node/.style={font=\scriptsize, align=center}]
        % Hauteur des blocs augmentée à 1.5cm pour tenir sur 2 lignes
        \draw[fill=SolanaBlue!20] (0,0) rectangle (8,1.5) node[midway] {\textbf{FONDATIONS}\\(S1-S8)};
        \draw[fill=SolanaPurple!20] (8,0) rectangle (20,1.5) node[midway] {\textbf{TRACK SPÉCIALISÉ}\\(S9-S20)};
        \draw[fill=SolanaGreen!20] (20,0) rectangle (24,1.5) node[midway] {\textbf{CAPSTONE}\\(S21-S24)};
        \draw[fill=MF_Gold!20] (24,0) rectangle (28,1.5) node[midway] {\textbf{CAREER}\\(S25-S28)};
        
        % Descriptions sous les blocs
        \node[below=2pt] at (4,0) {\tiny Rust, Crypto, CS};
        \node[below=2pt] at (14,0) {\tiny Solana/Anchor ou EVM/Solidity};
        \node[below=2pt] at (22,0) {\tiny Build \& Audit};
        \node[below=2pt] at (26,0) {\tiny Interview Prep};
    \end{tikzpicture}
    \caption{Timeline Macro du Cursus}
\end{figure}

\begin{center}\colorbox{SolanaBlue!20}{\parbox{\textwidth}{\centering \Large \textbf{NIVEAU 1 : PISCINE \& FONDATIONS (S1-S8)}}}\end{center}

\section{TRONC COMMUN : LA FORGE (S1-S8)}

\begin{table}[h]
    \caption{Synthèse Phase 0 \& 1}
    \centering
    \small
    \rowcolors{2}{SolanaBlue!5}{white}
    \begin{tabularx}{\textwidth}{l X X l}
        \toprule
        \textbf{Sem.} & \textbf{Focus Technique} & \textbf{Livrable Pivot} & \textbf{Gate Qualité} \\
        \midrule
        S1 & OS \& Git Internals & Réplique `ls -la` en Rust & Git Clean \\
        S2 & Memory Safety & Custom Allocator & No Leaks \\
        S3 & Concurrence & HTTP Server Multi-thread & Benchmarks \\
        S4 & Cryptographie & CLI Wallet (Ed25519) & Signatures valides \\
        \midrule
        S5 & Web3 Protocol & Architecture Diagram & C4 Model \\
        S6 & Consensus & Simulation PoS (Python/Rust) & Slashing rules \\
        S7 & Tokenomics & Whitepaper d'un DEX & Math verified \\
        S8 & Wallet Interaction & Connect Wallet (React) & UX smooth \\
        \midrule
        \multicolumn{4}{l}{\textbf{INTÉGRATION IA \& SÉCURITÉ OFFENSIVE}} \\
        S10 & AI-Assisted Eng. & Refactoring avec Windsurf & Audit IA vs Humain \\
        S26 & War Room & Simulation Hack & Post-Mortem \\
        \bottomrule
    \end{tabularx}
\end{table}

\begin{techBox}{La Stack du Vainqueur}
    \begin{itemize}
        \item \textbf{Rust} : Langage système, sécurité mémoire garantie sans Garbage Collector.
        \item \textbf{Anchor} : Framework de développement Solana qui sécurise et accélère le code.
        \item \textbf{Solidity} : Langage historique des Smart Contracts (EVM).
        \item \textbf{Foundry} : Outil de test et déploiement Ethereum écrit en Rust.
    \end{itemize}
\end{techBox}

\subsection{Détail des Semaines Critiques}

\subsubsection{Semaine 1 : Ingénierie Système}
\begin{kpiBox}{S1 : Git Internals}
    \textbf{Objectif :} Manipuler les blobs/trees Git en Rust sans `git`. \\
    \textbf{Livrable :} Outil CLI `my-git`. \\
    \textbf{Politique IA :} \textcolor{red}{\textbf{INTERDITE}}.
\end{kpiBox}

\subsubsection{Semaine 4 : Cryptographie}
\begin{kpiBox}{S4 : Primitives Crypto}
    \textbf{Objectif :} Implémenter SHA-256 et Ed25519 (Signatures). \\
    \textbf{Livrable :} CLI Wallet. \\
    \textbf{Politique IA :} \textcolor{red}{\textbf{INTERDITE}}.
\end{kpiBox}

\newpage
\begin{center}\colorbox{SolanaPurple!20}{\parbox{\textwidth}{\centering \Large \textbf{NIVEAU 2 : SPÉCIALISATION (S9-S20)}}}\end{center}

\section{TRACKS SPÉCIALISÉS (A/B/C)}
\textit{Le détail des tracks (Solana, EVM, Product) est disponible dans les chapitres dédiés.}

\newpage
\begin{center}\colorbox{SolanaGreen!20}{\parbox{\textwidth}{\centering \Large \textbf{EXTENSIONS \& OPTIONS}}}\end{center}

\section{Modules de Diversification (Electifs)}

Pour les étudiants souhaitant élargir leur spectre technique, RBK 2.0 propose des modules intensifs accessibles en parallèle ou post-cursus.

\subsection{Module ZK : Zero-Knowledge Proofs (8 semaines)}
\begin{itemize}
    \item \textbf{Contenu :} Arithmétisation, R1CS, Plonk, Langage Noir et Circom.
    \item \textbf{Projet :} Concevoir un mélangeur de tokens (Mixer) compliant (Privacy Pools).
    \item \textbf{Pré-requis :} Niveau Mathématique A+ (Algèbre linéaire).
\end{itemize}

\subsection{Module DePIN : Decentralized Physical Infra (6 semaines)}
\begin{itemize}
    \item \textbf{Contenu :} Helium Network, Filecoin, IoT Integration, Proof of Coverage.
    \item \textbf{Projet :} Déployer un réseau de capteurs LoRaWAN incentivé par token.
\end{itemize}

\subsection{Module Cross-Chain & Interop (4 semaines)}
\begin{itemize}
    \item \textbf{Contenu :} Wormhole, LayerZero, Axelar. Design de messages asynchrones.
    \item \textbf{Projet :} Bridge NFT Solana $\leftrightarrow$ Ethereum.
\end{itemize}
