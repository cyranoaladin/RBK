% ==============================================================================
% Annexe E : Cockpit RBK (Dashboard de Suivi)
% ==============================================================================
\chapter{ANNEXE — COCKPIT RBK (DASHBOARD ÉTUDIANT)}
\label{annexe:cockpit}

Le "Cockpit" (`cockpit.rbk.tn`) est l'outil central de pilotage de la performance étudiante. Il agrège les données techniques, comportementales et de santé mentale.

\section{Vue d'Ensemble}

\begin{techBox}{Fonctionnalités Clés}
\begin{itemize}
    \item \textbf{Suivi Temps Réel :} Synchronisation avec l'API GitHub.
    \item \textbf{Seniority Tracking :} Évolution sur la grille 0 à 4.
    \item \textbf{Santé Mentale :} Remontée des "Wellness Checks" hebdomadaires.
    \item \textbf{Employability Score :} Calculé pour l'accès au Job Board.
\end{itemize}
\end{techBox}

\section{Suivi des Compétences (Seniority Matrix)}

Visualisation radar des 5 axes de compétences :
\begin{enumerate}
    \item \textbf{Coding Standard} : Qualité, Linter, Types.
    \item \textbf{Architecture} : Modularité, Scalabilité, Patterns.
    \item \textbf{Testing} : Coverage, Fuzzing, Invariants.
    \item \textbf{Security} : Audit mindset, Threat modeling.
    \item \textbf{Collaboration} : Code Review, Documentation, Communication.
\end{enumerate}

\begin{figure}[h]
    \centering
    \begin{tikzpicture}
        \foreach \angle/\label in {90/Code, 162/Arch, 234/Test, 306/Secu, 18/Collab} {
            \draw[gray!30] (0,0) -- (\angle:3cm);
            \node at (\angle:3.3cm) {\small \label};
        }
        \draw[gray!30] (0,0) circle (1cm);
        \draw[gray!30] (0,0) circle (2cm);
        \draw[gray!30] (0,0) circle (3cm);
        
        \draw[SolanaGreen, thick, fill=SolanaGreen!20, opacity=0.7] 
        (90:2.5) -- (162:2.0) -- (234:2.8) -- (306:1.5) -- (18:2.9) -- cycle;
        \node[color=SolanaGreen, font=\bfseries] at (0,-3.5) {Profil Cible (Level 3)};
    \end{tikzpicture}
    \caption{Radar de Compétences (Cockpit View)}
\end{figure}

\section{Alerting & Monitoring (Anti-Burnout)}

Le système analyse les signaux faibles pour prévenir le décrochage :
\begin{itemize}
    \item \textbf{Signal GitHub} : Absence de commit > 3 jours (Orange).
    \item \textbf{Signal Discord} : Silence radio > 48h (Orange).
    \item \textbf{Signal Wellness} : Score self-reported < 2/5 (Rouge).
\end{itemize}

\section{Interface Employeur (Talent Pool)}
Les partenaires recruteurs accèdent à une vue filtrée ("Anonymized Mode" ou "Full Profile" avec consentement) :
\begin{itemize}
    \item \textbf{Badge Vérifié} : SBT Link.
    \item \textbf{Top Projects} : Liens directs vers les repos "Golden".
    \item \textbf{Soft Skills} : Note de collaboration et fiabilité (Ponctualité).
\end{itemize}
