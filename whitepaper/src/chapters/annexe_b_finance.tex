\chapter{MODÈLE FINANCIER DÉTAILLÉ}

Le modèle financier RBK 2.0 repose sur une approche **Hybride** robuste, privilégiant la liquidité immédiate via les frais Upfront tout en conservant un potentiel d'upside significatif via l'ISA pour les Top Talents.

\section{Hypothèses Structurantes}

Nous retenons le **Scénario Hybride** comme base du Business Plan.

\begin{center}
    \textbf{Structure de Revenus (Cible)} \\
    \small
    \rowcolors{2}{gray!5}{white}
    \begin{tabularx}{\textwidth}{|l|X|l|}
        \hline
        \textbf{Flux} & \textbf{Description} & \textbf{Part du Volume} \\ \hline
        \textbf{Upfront (B2C)} & Paiement direct par les étudiants (Niveaux 1, 2, 3 ou Pack). Assure le BFR immédiat. & \textbf{60\%} des étudiants \\ \hline
        \textbf{ISA (Différé)} & Paiement différé 15\% sur 36 mois. Option réservée aux profils "Top Potential" (N3/Pack). & \textbf{30\%} des étudiants \\ \hline
        \textbf{Services (B2B)} & Formation d'employés et placement, payé par les entreprises (Sponsoring/Hiring). & \textbf{10\%} des revenus \\ \hline
    \end{tabularx}
\end{center}

\section{Paramètres ISA \& Cash Drag}

L'option ISA introduit un décalage de trésorerie ("Cash Drag") que notre modèle anticipe :
\begin{itemize}
    \item \textbf{Taux de Placement :} Hypothèse conservatrice de **85\%** des étudiants ISA placés à 6 mois.
    \item \textbf{Délai de Paiement :} 
    \begin{itemize}
        \item Formation : 6 mois (Cycle complet).
        \item Recherche d'emploi : 3 mois (Moyenne).
        \item \textbf{Premier versement ISA :} À M+10 après le démarrage.
    \end{itemize}
    \item \textbf{Impact :} Les revenus ISA de l'Année 1 sont quasi nuls (amorçage). Ils deviennent significatifs en Année 2 (effet cumulatif des cohortes précédentes).
\end{itemize}

\section{Modèle Tarifaire (Base de calcul)}
Pour les besoins de la modélisation :
\begin{itemize}
    \item \textbf{Panier Moyen Upfront :} Estimé à \textbf{12 000 TND} (Mix pondéré : N1 seul, N1+N2 et Pack Complet Upfront).
    \item \textbf{Revenu Moyen ISA :} Estimé à \textbf{18 900 TND} (basé sur salaire moyen net 3 500 TND sur 36 mois : 525 x 36).
    \item \textbf{Marge Nette par Étudiant :} Cible > 35\% après coûts mentors et infrastructure.
\end{itemize}

\section{Unit Economics (Par Étudiant)}
\begin{center}
    \small
    \begin{tabular}{|l|r|l|}
        \hline
        \textbf{Poste} & \textbf{Montant (TND)} & \textbf{Note} \\ \hline
        \textbf{Revenu Moyen (A)} & \textbf{12 000} & Mix Upfront/ISA pondéré \\ \hline
        Mentorat (Variable) & (1 500) & Ratio expert 1:12 \\ \hline
        Infrastructure & (300) & Serveurs, SaaS, Licences \\ \hline
        Acquisition (CAC) & (500) & Marketing digital \\ \hline
        \textbf{Total Coût Variable (B)} & \textbf{(2 300)} & COGS \\ \hline
        \textbf{Marge Contributive (A-B)} & \textbf{9 700} & \textbf{80\% de Marge Brute} \\ \hline
    \end{tabular}
\end{center}

\section{Rentabilité \& Seuil}
Le seuil de rentabilité opérationnelle (Breakeven) est atteint dès que le volume dépasse **40 étudiants payants / an** (Niveau 1+), ce qui sécurise la structure indépendamment des succès ISA.
