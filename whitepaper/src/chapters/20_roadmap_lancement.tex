% ========================================

\chapter{{FEUILLE DE ROUTE~: LE PLAN DE LANCEMENT (90 JOURS)}}

Le plan est divisé en trois phases de 30 jours, chacune avec des livrables non négociables et des jalons de décision pour le CEO.

\section{MOIS 1~: CADRAGE, ALLIANCE \& ÉQUIPE NOYAU (J0 - J30)}

L'objectif est de verrouiller la structure juridique, financière et partenariale.

\subsection{Validation \& Cadrage Stratégique}
\begin{itemize}
    \item Présentation et validation finale du Business Plan par le CEO et le conseil d'administration.
    \item Arbitrage sur le modèle de financement étudiant (ISA vs Paiement classique vs Prêts bancaires).
\end{itemize}

\subsection{Constitution de l'Alliance Écosystémique}
\begin{itemize}
    \item \textbf{Action Clé :} Signature d'un MOU (Protocole d'accord) avec la Solana Foundation pour l'accréditation «~Educational Partner~».
    \item Dépôt de candidature pour opérer le chapitre officiel Superteam Tunisia.
    \item Négociation avec des universités d'élite (ESPRIT, MSB, Polytech) pour l'accréditation académique ou des passerelles de fin d'études.
\end{itemize}

\subsection{Recrutement de l'Équipe Pilote}
\begin{itemize}
    \item Recrutement du Lead Instructor / Head of Curriculum (Expert Rust/Solana).
    \item Désignation d'un Responsable Administratif \& Logistique dédié au Studio.
    \item Identification du pool de mentors internationaux (Guest Lecturers) via le réseau Superteam (UK, Allemagne, UAE).
\end{itemize}

\section{MOIS 2~: PRODUCTION DE L'ARSENAL \& INFRASTRUCTURE (J31 - J60)}

L'objectif est de bâtir l'infrastructure technique et les contenus de référence.

\subsection{Ingénierie Pédagogique (Les «~Golden Templates~»)}
\begin{itemize}
    \item Rédaction détaillée des syllabus pour le Tronc Commun et les Tracks A (Solana) \& B (EVM).
    \item Création des dépôts GitHub de référence (repos «~or~») incluant les architectures de base, les tests de sécurité et les pipelines CI/CD.
    \item Élaboration des «~Incident Drills~» (simulations de hacks pour les exercices du vendredi).
\end{itemize}

\subsection{Mise en place du Cockpit Technique}
\begin{itemize}
    \item Configuration des accès aux Nodes RPC premium (Helius pour Solana, Alchemy/Infura pour EVM).
    \item Acquisition des licences pour les outils d'IA (Cursor, Windsurf) et de simulation (Machinations.io).
    \item Installation du LMS (Learning Management System) et du serveur Discord comme hub de communication principal.
\end{itemize}

\subsection{Lancement Commercial \& Marketing}
\begin{itemize}
    \item Mise en ligne du site web dédié au programme.
    \item Lancement de la campagne marketing «~Elite Only~» sur LinkedIn et Twitter (X).
    \item Organisation du premier «~Tunisian Web3 Builder Meetup~» pour générer des leads qualifiés.
\end{itemize}

\section{MOIS 3~: SÉLECTION \& LANCEMENT «~PROMO ALPHA~» (J61 - J90)}

L'objectif est de filtrer les talents et de démarrer l'immersion.

\subsection{Processus de Sélection d'Élite}
\begin{itemize}
    \item Tests techniques de pré-requis (JS/TS intensif).
    \item Entretiens de motivation pour évaluer la pensée systémique et l'autonomie.
    \item La «~Piscine~» Rust~: Lancement de la phase de filtrage intensif de 4 semaines sans IA pour les 20 candidats présélectionnés.
\end{itemize}

\subsection{Finalisation de la Cohorte}
\begin{itemize}
    \item Sélection finale de la Cohorte Alpha (15 à 20 profils maximum pour garantir l'excellence).
    \item Signature des contrats (incluant les clauses ISA le cas échéant).
    \item Onboarding sur Superteam Earn pour que les étudiants voient les premières opportunités de revenus dès le début.
\end{itemize}

\subsection{Kick-off Opérationnel}
\begin{itemize}
    \item Cérémonie de lancement en présence de partenaires de l'écosystème.
    \item Début de la Phase 1 (Fondations \& Mentalité On-chain).
\end{itemize}

\section{RÉCAPITULATIF DES JALONS CLÉS (MILESTONES)}

\begin{table}[ht]
\centering
\begin{tabularx}{\textwidth}{c|X|X}
\midrule
\rowcolor{SolanaPurple!20} \textbf{Délai} & \textbf{Jalon (Milestone)} & \textbf{Impact} \\
\midrule
J+15 & MOU Solana Foundation signé & Crédibilité internationale immédiate. \\
\midrule
J+30 & Équipe pédagogique complète & Capacité de production activée. \\
\midrule
J+45 & Golden Templates livrés & Standard de qualité «~Senior-by-Design~» fixé. \\
\midrule
J+60 & 100 leads qualifiés générés & Sécurité du taux de remplissage. \\
\midrule
J+75 & Fin de la «~Piscine~» Rust & Cohorte d'élite validée. \\
\midrule
J+90 & Lancement officiel Promo Alpha & Début de la transformation de RBK. \\
\midrule
\end{tabularx}
\caption{Jalons Clés du Plan de Lancement}
\end{table}

\begin{strategieBox}{Annotation Stratégique}
Ce plan de 90 jours est agressif mais réaliste. Il repose sur l'utilisation intensive des ressources existantes de RBK (locaux, réseau alumni) et sur l'apport d'expertise Web3 externe pour l'ingénierie de contenu.
\end{strategieBox}

% ========================================
% CHAPITRE 15~: ÉLÉMENTS DE DIFFÉRENCIATION
