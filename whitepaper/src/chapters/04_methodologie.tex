\chapter{MÉTHODOLOGIE CYBORG 2.0}
\label{chap:methodologie}

\section{Philosophie Pédagogique : Intégration du Bien-être}

La méthodologie RBK 2.0 ne se contente pas de former des techniciens ; elle forge des \textit{athlètes cognitifs}. Conscients de la charge mentale intense imposée par l'apprentissage du développement blockchain (Rust, Zero-Knowledge Proofs, audits de sécurité), nous avons intégré une dimension \textbf{santé mentale et résilience} au cœur même du curriculum.
\marginpar{\footnotesize \faHeartbeat\ \textbf{Priorité Absolue} : La santé mentale de l'étudiant est notre actif le plus précieux.}

\begin{figure}[h]
    \centering
    \begin{tikzpicture}[
        scale=0.9,
        every node/.style={scale=0.9},
        day/.style={rectangle, draw=SolanaPurple, fill=white, rounded corners, minimum width=2.5cm, minimum height=3cm, align=center, drop shadow},
        arrow/.style={->, >=Stealth, thick, SolanaBlue},
        label/.style={font=\bfseries\small, color=BaseDark}
    ]
        % Nodes
        \node[day] (lundi) at (0,0) {\textbf{LUNDI}\\\footnotesize Planning\\Stratégique\\\faBrain\ Focus};
        \node[day] (mardi) at (3.5,0) {\textbf{MARDI}\\\footnotesize Deep Work\\(IA Assisted)\\\faLaptopCode\ Dev};
        \node[day] (mercredi) at (7,0) {\textbf{MERCREDI}\\\footnotesize Pair\\Programming\\\faUserFriends\ Collab};
        \node[day] (jeudi) at (10.5,0) {\textbf{JEUDI}\\\footnotesize Audit \&\\Review\\\faBug\ Quality};
        \node[day] (vendredi) at (14,0) {\textbf{VENDREDI}\\\footnotesize Débrief \&\\Resilience\\\faHeartbeat\ Santé};

        % Arrows
        \draw[arrow] (lundi) -- (mardi);
        \draw[arrow] (mardi) -- (mercredi);
        \draw[arrow] (mercredi) -- (jeudi);
        \draw[arrow] (jeudi) -- (vendredi);

        % Brace
        \draw[decorate, decoration={brace, amplitude=10pt}, thick, color=SolanaGreen] (0,1.7) -- (14,1.7) node[midway, above=15pt, color=SolanaGreen, font=\bfseries] {CYCLE HEBDOMADAIRE "SUSTAINABLE PERFORMANCE"};

    \end{tikzpicture}
    \caption{Le Cycle Hebdomadaire RBK 2.0}
\end{figure}

\subsubsection{Le Contrat de Performance Durable}
Nous imposons un cadre strict pour éviter le surmenage :
\begin{enumerate}
    \item \textbf{Deep Work Timeboxé :} Maximum 6 heures de code pur par jour. Au-delà, la productivité et la qualité du code chutent (bugs).
    \item \textbf{No-Code Weekend :} Interdiction de pousser du code sur GitHub du samedi 12h au lundi 8h (sauf Hackathon exceptionnel).
    \item \textbf{Rituel de Décompression :} Session de sport ou méditation obligatoire le vendredi après-midi.
\end{enumerate}

\subsubsection{Cadre d'Usage de l'IA (Cyborg Policy)}
L'IA est un levier, pas une béquille. Son usage est régulé :

\paragraph{Niveau 0 (Piscine) : Interdiction Totale}
L'étudiant doit développer ses modèles mentaux sans assistance. Copilot est désactivé. Toute détection de code généré entraîne une disqualification.

\paragraph{Niveau 1+ (Cursus) : Assistance Supervisée}
L'IA est autorisée pour :
\begin{itemize}
    \item Générer des tests unitaires (TDD).
    \item Expliquer des messages d'erreur obscurs.
    \item Produire du boilerplate (structs, imports).
\end{itemize}
Elle est \textbf{interdite} pour :
\begin{itemize}
    \item Résoudre l'exercice à la place de l'étudiant.
    \item Générer la logique core sans audit manuel ligne par ligne.
\end{itemize}

\section{Standardisation d'Ingénierie}

\subsection{Matrice de Séniorité (0-4)}
Pour objectiver la progression, nous utilisons une grille inspirée des "Engineering Ladders" des grandes tech.

\begin{table}[h]
    \caption{Seniority Matrix RBK}
    \centering
    \small
    \rowcolors{2}{gray!5}{white}
    \begin{tabularx}{\textwidth}{|l|l|X|}
    \hline
    \textbf{Niveau} & \textbf{Titre} & \textbf{Attendu} \\ \hline
    \textbf{0} & Aspiring & Suit les tutos, code fragile, a besoin d'aide constante. \\ \hline
    \textbf{1} & Junior & Code fonctionnel, tests basiques, autonome sur tâches simples. \\ \hline
    \textbf{2} & Mid-Level & Architecture propre, tests E2E, commence à reviewer les autres. \\ \hline
    \textbf{3} & Senior & Design patterns, optimisation gas/mémoire, mentorat actif. \\ \hline
    \textbf{4} & Staff+ & Vision système, sécurité offensive, contribution Open Source. \\ \hline
    \end{tabularx}
\end{table}

\subsection{Standardisation des Repos GitHub}
Pas de "code spaghetti". Tous les projets étudiants doivent respecter la structure de l'organisation \texttt{rbk-studio}.

\subsubsection{Workflow CI/CD Standard (YAML)}
Chaque repo doit obligatoirement inclure ce workflow dans `.github/workflows/security.yml` :

\begin{lstlisting}[language=yaml, caption=RBK Security Workflow Standard]
name: RBK Security Gate
on: [push, pull_request]

jobs:
  audit:
    runs-on: ubuntu-latest
    steps:
      - uses: actions/checkout@v3
      - name: Install Audit Tools
        run: |
          cargo install cargo-audit
          npm install -g solhint
      - name: Static Analysis
        run: |
          cargo audit
          solhint 'contracts/**/*.sol'
      - name: Fuzzing Check
        run: ./scripts/fuzz.sh --duration 300
\end{lstlisting}

\subsubsection{Script d'Audit Automatisé (Bash)}
Exemple de script `scripts/audit.sh` à inclure :

\begin{lstlisting}[language=bash, caption=Pre-Commit Audit Script]
#!/bin/bash
set -e
echo "🔎 Starting RBK Pre-Commit Audit..."

# 1. Check for Sensitive Keys
if grep -r "PRIVATE_KEY" .; then
    echo "❌ CRITICAL: Private Key found in code!"
    exit 1
fi

# 2. Run Tests
echo "🧪 Running Tests..."
anchor test

# 3. Verify Formatting
echo "🎨 Checking Style..."
cargo fmt -- --check

echo "✅ Audit Passed. Ready to Push."
\end{lstlisting}

\begin{techBox}{Organisation rbk-studio (Structure Complète)}
rbk-studio/
├── .github/ (Config globale, Code of Conduct)
├── templates/
│   ├── solana-anchor-template/ (Reference Standard)
│   └── evm-foundry-template/
├── tools/
│   └── rbk-cli/ (Scaffolding tool)
└── docs-commons/
    ├── ADR-template.md
    ├── threat-model-template.md
    └── code-review-checklist.md
\end{techBox}

\begin{techBox}{Template Standard (Solana/Anchor)}
solana-anchor-template/
├── programs/
│   └── my-program/ (Logique Business)
├── tests/ (Typescript E2E)
├── migrations/ (Scripts de déploiement)
├── .github/workflows/ (CI/CD Obligatoire)
│   ├── ci.yml (Lint/Test/Build)
│   └── security.yml (Audit auto)
└── docs/ (ADR, Threat Model, Runbook)
\end{techBox}

\subsubsection{Règles d'Or GitHub}
\begin{enumerate}
    \item \textbf{README Parfait :} Badges de build, intro claire, quickstart en 1 ligne commande.
    \item \textbf{Documentation :} Dossier `docs/` avec \textbf{ADR} (Architecture Decision Records).
    \item \textbf{CI/CD :} Pas de merge sur main sans que la CI soit verte (Tests + Lint).
\end{enumerate}

\section{La « Piscine » Rust : Programme Pré-Piscine}

Pour maximiser les chances de succès et réduire le taux d'abandon, RBK 2.0 intègre une phase préparatoire structurée.

\subsubsection{Objectif : Filtrer la Rigueur}
La Piscine Rust n'évalue pas le niveau informatique initial (nous acceptons les débutants brillants), mais la capacité d'apprentissage rapide et la résilience à l'échec. C'est un test de caractère.

\subsubsection{Rubrique d'Évaluation (Scoring)}
Nous utilisons une grille précise pour objectiver la sélection :

\begin{table}[h]
    \caption{Critères de Sélection Pré-Piscine}
    \centering
    \small
    \rowcolors{2}{SolanaBlue!5}{white}
    \begin{tabularx}{\textwidth}{l X c c}
        \toprule
        \textbf{Critère} & \textbf{Description} & \textbf{Poids} & \textbf{Seuil Min.} \\
        \midrule
        \textbf{Rustlings} & Complétion des 80 exercices de syntaxe & 30\% & 100\% \\
        \textbf{Algo (Codewars)} & Résolution de problèmes logiques (Katas) & 30\% & Rank 5kyu \\
        \textbf{Git Hygiene} & Qualité des commits (Atomicité, Messages) & 20\% & Pro \\
        \textbf{Discipline} & Régularité des pushs (Green Dots) & 20\% & Quotidien \\
        \bottomrule
    \end{tabularx}
\end{table}

\subsubsection{Anti-Triche et Preuve de Travail}
Pour garantir que c'est bien l'étudiant qui code :
\begin{itemize}
    \item \textbf{Entretiens Flash :} Le mentor demande d'expliquer une ligne de code aléatoire en direct.
    \item \textbf{Live Coding :} Une épreuve finale surveillée (proctored) sans IA.
    \item \textbf{Analyse Stylométrique :} Détection des changements brusques de style de code (indiquant un copier-coller).
\end{itemize}

\section{Protocole Anti-Burnout}

Nous avons industrialisé la protection de nos étudiants via un protocole strict.

\subsubsection{Monitoring Hebdomadaire}
Chaque vendredi, les étudiants remplissent un "Wellness Check" anonymisé de 5 questions :
\begin{enumerate}
    \item Qualité du sommeil (1-5).
    \item Niveau de stress perçu (1-5).
    \item Sentiment de compétence (Impostor Syndrome) (1-5).
\end{enumerate}

\subsubsection{Seuils et Escalade (Traffic Light Protocol)}

\begin{table}[h]
    \caption{Matrice d'Intervention Santé Mentale}
    \centering
    \small
    \rowcolors{2}{red!5}{white}
    \begin{tabularx}{\textwidth}{l X l l}
        \toprule
        \textbf{Zone} & \textbf{Critère Déclencheur} & \textbf{Action Immédiate} & \textbf{Responsable} \\
        \midrule
        \textbf{\textcolor{green}{VERT}} & Score > 4/5 & Rien à signaler & Mentor \\
        \textbf{\textcolor{orange}{ORANGE}} & Score < 3/5 ou Retard livrables & Entretien 1-on-1 & Student Success \\
        \textbf{\textcolor{red}{ROUGE}} & Score < 2/5 ou "Panic Attack" & \textbf{Arrêt forcé 48h} & Head of Ed \\
        \bottomrule
    \end{tabularx}
\end{table}

\subsubsection{Plan de Remédiation}
En cas de zone rouge persistante, nous activons la "Pause Fusible" :
\begin{itemize}
    \item \textbf{Semaine Off :} L'étudiant coupe tout écran pendant 7 jours sans pénalité.
    \item \textbf{Rattrapage :} Il réintègre la cohorte avec un plan allégé ou bascule sur la cohorte suivante (Roll-over) si nécessaire.
\end{itemize}

\begin{figure}[h]
    \centering
    \begin{tikzpicture}[scale=0.8, node distance=2cm]
        \node[draw, rectangle, rounded corners] (quest) {Questionnaire Hebdo};
        \node[draw, diamond, aspect=2, below of=quest] (score) {Scoring Auto};
        \node[draw, rectangle, fill=green!20, below left of=score, xshift=-1cm] (ok) {Continue};
        \node[draw, rectangle, fill=orange!20, below of=score] (warn) {Alerte Mentor};
        \node[draw, rectangle, fill=red!20, below right of=score, xshift=1cm] (stop) {STOP URGENCE};

        \draw[->] (quest) -- (score);
        \draw[->] (score) -| node[near start, above] {> 4} (ok);
        \draw[->] (score) -- node[near start, right] {2 - 4} (warn);
        \draw[->] (score) -| node[near start, above] {< 2} (stop);
    \end{tikzpicture}
    \caption{Algorithme de Décision Anti-Burnout}
\end{figure}
