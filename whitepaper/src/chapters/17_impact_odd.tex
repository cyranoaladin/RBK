\chapter{IMPACT SOCIAL \& ALIGNEMENT ODD}

RBK 2.0 est une entreprise à mission. Notre but est de transférer de la richesse du PIB mondial (Web3) vers l'économie locale tunisienne et africaine.

\section{Contribution aux Objectifs de Développement Durable (ONU)}

\begin{itemize}
    \item \textbf{ODD 4 : Éducation de Qualité.} Nous démocratisons l'accès à une formation d'élite (niveau Ivy League) sans barrière financière grâce à l'ISA.
    \item \textbf{ODD 8 : Travail Décent et Croissance Économique.} Nous créons des emplois à haute valeur ajoutée, exportateurs de services, et rémunérés en devises fortes (via le statut local adéquat).
    \item \textbf{ODD 9 : Industrie, Innovation et Infrastructure.} Nous formons les architectes de l'infrastructure financière de demain.
\end{itemize}

\section{Indicateurs de Performance Sociale}

\subsection{1. Inclusion des Femmes dans la Tech}
Le Web3 souffre d'un déficit de diversité criant. RBK 2.0 met en place des mesures proactives :
\begin{itemize}
    \item \textbf{Bourses "Women in Web3" :} Le coût Upfront est réduit de 50\% pour les candidates validant la Piscine (financé par partenaires).
    \item \textbf{Objectif 2026 :} Atteindre 30\% de femmes par cohorte (vs 5\% moyenne secteur).
\end{itemize}

\subsection{2. Décentralisation Régionale}
Le talent est partout, les opportunités sont à la capitale.
\begin{itemize}
    \item \textbf{Recrutement National :} Roadshow dans les universités de l'intérieur (Sfax, Gabès, Gafsa).
    \item \textbf{Hébergement :} Partenariats avec des foyers pour faciliter l'installation à Tunis durant les 4 mois intensifs.
\end{itemize}

\subsection{3. Empreinte Carbone et Compensation}
La blockchain est perçue comme polluante. RBK nuance et agit :
\begin{itemize}
    \item \textbf{Choix Technologique :} Solana est une chaîne Proof-of-Stake dont une transaction consomme moins qu'une requête Google (0.0005 kWh).
    \item \textbf{Compensation :} Nous nous engageons à compenser 100\% de l'empreinte carbone de l'école (serveurs + clim + déplacements staff) via l'achat de crédits carbone certifiés on-chain (ex: Toucan Protocol).
\end{itemize}
