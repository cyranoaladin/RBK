\chapter[BUSINESS PLAN \& STRATÉGIE DE CROISSANCE]{BUSINESS PLAN \& STRATÉGIE DE CROISSANCE}
\label{chap:business_plan}

\section{Hypothèses de Revenus & Salaires}
\label{sec:revenus_salaires}

\begin{tcolorbox}[colback=yellow!10, colframe=orange!80!black, title=\textbf{Note sur les Données de Marché}]
Les chiffres ci-dessous sont des projections basées sur l'arbitrage "Salaire Global / Coût de Vie Local". 
Ils supposent :
\begin{enumerate}
    \item Une capacité à travailler en **Anglais** (marché international).
    \item L'obtention du statut **ETE** (optimisation fiscale nette).
    \item Une facturation en devises (EUR/USD) convertie en TND.
\end{enumerate}
Le salaire moyen local "Web2 Senior" (environ 2.5k TND) sert de plancher, pas de cible.
\end{tcolorbox}

\section{Modèle Économique Hybride (60/30/10)}

Pour maximiser la résilience, RBK 2.0 adopte un mix revenus stratégique :
\begin{itemize}
    \item \textbf{60\% Upfront (Cashflow immédiat)} : Sécurise les opex (salaires, locaux).
    \item \textbf{30\% ISA (Cashflow futur)} : Aligne les intérêts ("Skin in the Game") et crée un fond de roulement exponentiel.
    \item \textbf{10\% B2B (Marge pure)} : Valorise l'expertise pédagogique auprès des entreprises (Formation continue).
\end{itemize}

\begin{figure}[H]
    \centering
    \begin{tikzpicture}
        \draw[fill=SolanaPurple!20] (0,0) circle (2cm);
        \node at (0,0.5) {\textbf{Upfront (60\%)}};
        \node at (0,-0.5) {\footnotesize Trésoimmédiate};
        
        \draw[fill=SolanaBlue!20] (3,0) circle (1.5cm);
        \node at (3,0.5) {\textbf{ISA (30\%)}};
        \node at (3,-0.5) {\footnotesize Upside (M+10)};
        
        \draw[fill=SolanaGreen!20] (5.5,0) circle (1cm);
        \node at (5.5,0.3) {\textbf{B2B (10\%)}};
        \node at (5.5,-0.3) {\footnotesize Corporate};
    \end{tikzpicture}
    \caption{Mix Revenus Stratégique}
\end{figure}

\section{Unit Economics \& CAC}
\begin{table}[H]
    \centering
    \small
    \rowcolors{2}{gray!5}{white}
    \begin{tabularx}{\textwidth}{|l|r|X|}
    \hline
    \textbf{Métrique} & \textbf{Valeur} & \textbf{Hypothèses} \\ \hline
    \textbf{LTV (Lifetime Value)} & $\approx$ 12 500 TND & Mix N1+N2+N3 moyen par étudiant inscrit. \\ \hline
    \textbf{CAC (Coût Acquisition)} & 450 TND & Blended (Ads + Events + Referral). \\ \hline
    \textbf{Coût Mentorat} & 1 500 TND / étudiant & Ratio 1:12, Mentors payés à l'heure + variable. \\ \hline
    \textbf{Marge Brute} & $\approx$ 80\% & Business model à fort levier (Digital). \\ \hline
    \textbf{Point Mort (Breakeven)} & \textbf{40 Étudiants/an} & Seuil de rentabilité structurelle basse. \\ \hline
    \end{tabularx}
\end{table}

\section{Scénarios de Sensibilité (Stress Tests)}
Notre modèle a été éprouvé contre des scénarios de crise pour valider sa robustesse.

\begin{table}[H]
    \caption{Impact EBITDA Année 2}
    \centering
    \small
    \begin{tabular}{|l|l|c|l|}
    \hline
    \textbf{Scénario} & \textbf{Détail} & \textbf{Impact} & \textbf{Réponse Stratégique} \\ \hline
    \textbf{"Crypto Winter"} & Chute demande Web3 & -120k & Pivot vers Rust/Systems Programming (Marché Embedded). \\ \hline
    \textbf{"Défaut ISA"} & Taux défaut 15\% & -45k & Activation Fonds Garantie + Recouvrement Légal. \\ \hline
    \textbf{"Low Conversion"} & N1$\to$N2 à 40\% & -90k & Réduction Opex (Mentors) + Focus B2B plus tôt. \\ \hline
    \end{tabular}
\end{table}

\section{Gestion du "Cash Drag" ISA}
L'ISA crée un décalage de trésorerie de 10 à 18 mois (Formation + Recherche Emploi + Franchise).
\textbf{Impact :} L'Année 1 est cash-negative sur le segment ISA.
\textbf{Solution :} Financement par CA Upfront et Capital d'amorçage (Levée ou Prêt d'honneur) pour couvrir le BFR de 200k TND.

\subsubsection{Tableau de Flux de Trésorerie (Forecast Mensuel)}
\begin{table}[H]
    \caption{Projection Cash Flow (Milliers TND)}
    \centering
    \small
    \rowcolors{2}{gray!5}{white}
    \begin{tabular}{|l|c|c|c|c|}
    \hline
    \textbf{Mois} & \textbf{Année 1 (Sem 1)} & \textbf{Année 1 (Sem 2)} & \textbf{Année 2} & \textbf{Année 3} \\ \hline
    \textbf{Encaissements} & 150 & 200 & 850 & 1 500 \\ \hline
    \textbf{Décaissements} & (180) & (150) & (550) & (900) \\ \hline
    \textbf{Solde Période} & \textbf{(30)} & \textbf{+50} & \textbf{+300} & \textbf{+600} \\ \hline
    \textbf{Cash Fin Période} & 170 (Post-Levée) & 220 & 520 & 1 120 \\ \hline
    \end{tabular}
\end{table}

\section{Financements et Partenariats Stratégiques}

Pour accélérer sans diluer le capital, RBK active les leviers non-dilutifs :

\subsection{1. Écosystème Web3 (Grants)}
\begin{itemize}
    \item \textbf{Solana Foundation :} Demande de grant "Education" pour financer les serveurs et les bourses (Target: 50k\$).
    \item \textbf{Superteam :} Sponsoring des Hackathons de fin de cohorte (Prize pool).
\end{itemize}

\subsection{2. Bailleurs de Fonds Institutionnels}
\begin{itemize}
    \item \textbf{Union Européenne (Erasmus+ / Horizon Europe) :} Projets de mobilité des talents numériques Afrique-Europe.
    \item \textbf{Banque Africaine de Développement (BAD) :} Programme "Coding for Jobs".
\end{itemize}

\subsection{3. Modèle de Franchise (Scale Africa)}
Dès l'Année 3, le modèle "RBK in a Box" (LMS + Programme + Brand) sera proposé en franchise à des hubs technologiques au Sénégal et en Côte d'Ivoire.
\begin{itemize}
    \item \textbf{Modèle :} Revenue Share (20\% du CA Franchise).
    \item \textbf{Apport :} RBK fournit la plateforme et la certification SBT. Le partenaire gère le local et le sourcing.
\end{itemize}
