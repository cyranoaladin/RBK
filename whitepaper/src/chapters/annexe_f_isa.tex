% =========================
% ANNEXE ISA — Modèle propre
% =========================
\chapter{ANNEXE — Modèle ISA (Income Share Agreement)}

\section{Objet et Principes}
L’ISA est un mécanisme de financement \textbf{sélectif} destiné à aligner l’école et l’étudiant : l’étudiant ne paie que s’il dépasse un seuil de revenu, et l’école accepte un risque (non-emploi, variabilité, défaut). L’ISA est réservé aux profils validés \textbf{Top Talent}.

\section{Éligibilité (Gating)}
\begin{itemize}
    \item \textbf{Périmètre :} réservé au \textbf{parcours complet} (N1+N2+N3), avec dérogation exceptionnelle (sur dossier) pour admission directe en N3.
    \item \textbf{Quota :} nombre de places ISA limité par cohorte (ex : 30\% max) afin de préserver la trésorerie.
    \item \textbf{Sélection :} top performance (tests + discipline + livrables) + validation par comité (risques \& éthique).
\end{itemize}

\section{Définitions Normalisées (Net/Brut)}
Pour supprimer toute ambiguïté, les termes sont normalisés comme suit :

\paragraph{Revenu Net Mensuel (RNM).}
Montant \textbf{net} effectivement perçu et traçable pour un mois $m$, exprimé en TND :
\begin{itemize}
    \item \textbf{Emploi local :} net indiqué sur fiche de paie (après retenues).
    \item \textbf{Remote / freelance :} montant net crédité sur le compte bancaire (après frais de plateforme/banque), converti en TND au \textbf{taux mensuel} retenu (preuve : relevés).
\end{itemize}

\textbf{Définition (anti-ambiguïté) :} « Revenu net mensuel » désigne le montant net effectivement encaissé par l’étudiant (après charges sociales et impôts). En cas de revenus en devise, la conversion se fait au taux de référence du mois d’encaissement.

\paragraph{Seuil de Déclenchement.}
\textbf{3\,000 TND nets / mois}. Si $RNM(m) \le 3\,000$, alors paiement $= 0$.

\paragraph{Taux de Partage.}
\textbf{15\%} appliqué au \textbf{RNM} net mensuel encaissé (si > Seuil).

\paragraph{Cap (Plafond).}
Montant total maximal remboursable (toutes mensualités cumulées) : \textbf{CAP = 20\,000 TND}. Dès que le cumul atteint CAP, le contrat s’éteint.

\paragraph{Durée Maximale.}
\textbf{36 mensualités} de paiement effectif maximum. Une fenêtre d’extinction « drop-off » met fin au contrat au bout de \textbf{60 mois calendaires}, même si les 36 mensualités n’ont pas été atteintes (protection étudiant).

\section{Règles de Pause, Chômage, Variabilité}
\begin{itemize}
    \item \textbf{Pause automatique :} si $RNM(m) \le 3\,000$ (chômage ou revenu faible), paiement = 0.
    \item \textbf{Reprise :} dès que $RNM(m) > 3\,000$, le paiement reprend.
    \item \textbf{Variabilité :} aucun rattrapage sur les mois faibles ; le calcul est \textbf{mensuel}, indépendant.
\end{itemize}

\section{Cas Limites (Edge Cases) — Précisions Contractuelles}
\begin{enumerate}
    \item \textbf{RNM fluctuant autour du seuil :} si 2\,900 puis 3\,200, seul le mois à 3\,200 déclenche un paiement.
    \item \textbf{Plusieurs revenus :} RNM = somme des nets traçables (salaires + freelance) pour le mois.
    \item \textbf{Revenus en devise :} conversion en TND au taux mensuel convenu (source bancaire/BC), preuve par relevé.
    \item \textbf{Déclaration incomplète :} \textbf{Pénalités de retard (standard)} : 1\% par mois de retard sur le montant dû (calcul pro-rata), plafonné à 10\%. Aucune pénalité n’est appliquée durant une période de suspension (revenu < seuil / chômage). L’activation du recouvrement formel intervient uniquement après 90 jours de retard et après relances documentées.
    \item \textbf{Congé maladie / arrêt :} RNM baisse $\Rightarrow$ pause automatique (aucune pénalité).
    \item \textbf{Paiement anticipé :} option de clôture anticipée : l’étudiant peut solder le restant dû jusqu’au CAP.
    \item \textbf{Départ à l’étranger :} RNM calculé de la même manière (preuve par virements et relevés).
\end{enumerate}

\section{Exemples Chiffrés (Seuil 3\,000 net, Taux 15\%)}
\begin{center}
\small
\begin{tabular}{|l|r|r|l|}
\hline
\textbf{Scénario} & \textbf{RNM} & \textbf{Mensualité} & \textbf{Statut Final} \\
\hline
A. Junior local & 3\,500 TND & 525 TND & \textbf{Arrêt à 36 mois} (Total payé : 18\,900 TND < Cap) \\
\hline
B. Profil solide & 5\,000 TND & 750 TND & \textbf{Cap atteint} au 27ème mois (Total : 20\,000 TND) \\
\hline
C. Remote & 6\,000 TND & 900 TND & \textbf{Cap atteint} au 23ème mois (Total : 20\,000 TND) \\
\hline
D. Chômage & 0 TND & 0 TND & \textbf{Drop-off à 60 mois} (Paiement total : 0 TND) \\
\hline
E. Petit job & 2\,500 TND & 0 TND & \textbf{Drop-off à 60 mois} (Paiement total : 0 TND) \\
\hline
\end{tabular}
\end{center}
\textit{Note : Le contrat s'arrête dès que l'une des 3 conditions est atteinte : (1) Cap 20k payé, (2) 36 mensualités payées, ou (3) 60 mois écoulés.}
