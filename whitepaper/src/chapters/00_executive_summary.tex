% =========================
% EXECUTIVE SUMMARY (1 page)
% =========================
\chapter*{EXECUTIVE SUMMARY}
\addcontentsline{toc}{chapter}{Executive Summary}

\section*{Executive Summary (Investor-Style)}

\textbf{RBK 2.0} est un programme intensif \textbf{28 semaines} (format \textbf{stackable}) qui transforme des profils techniques motivés en \textbf{talents Web3 employables} sur des métiers \textbf{à forte barrière} (smart contracts, sécurité, product shipping). Le positionnement est \textbf{Senior-by-Design} : preuve par le code, exigence d’audit, culture de standards industriels.

\vspace{0.6em}
\textbf{Problème.} Le marché souffre d’une pénurie de profils capables de \textit{livrer} (shipping) et de \textit{sécuriser} (audit mindset). Les bootcamps « juniors » deviennent fragiles face à l’IA : RBK 2.0 monte en gamme vers l’architecture, les invariants et la sécurité.

\vspace{0.6em}
\textbf{Solution.} Un parcours en 3 niveaux \textbf{progressifs} (admissions directes possibles via test exigeant), chacun avec des livrables mesurables :
\begin{itemize}
    \item \textbf{Niveau 1 (Fondations)} : rigueur, Git, tests, mentalité on-chain (filtrage + discipline).
    \item \textbf{Niveau 2 (Spécialisation)} : track d’exécution (smart contracts / EVM) avec labs contrôlés.
    \item \textbf{Niveau 3 (Pro-Audit-Placement)} : capstones studio-grade, audit interne, employability pack.
\end{itemize}

\vspace{0.6em}
\textbf{Traction/Proof (KPI cibles, normalisés).}
\begin{itemize}
    \item \textbf{Format cohorte :} 20 apprenants / cohorte (qualité \& suivi).
    \item \textbf{Objectif placement :} \textbf{90\% à 6 mois} (cible interne ; obligation de moyens).
    \item \textbf{Cible revenu diplômé :} \textbf{3\,500 TND nets/mois (local)} ou \textbf{équivalent devise} (remote), exprimé en \textbf{TND nets équivalent} au taux mensuel de conversion.
\end{itemize}

\vspace{0.6em}
\textbf{Unit economics (cadre de lecture investisseur).}
\begin{itemize}
    \item \textbf{Prix public Pack Complet :} \textbf{18\,000 TND} (paiement échelonné possible).
    \item \textbf{Option ISA :} \textbf{15\%} au-delà d’un \textbf{seuil de 3\,000 TND nets/mois} (voir Annexe ISA), mécanisme réservé aux \textbf{Top Talents}.
    \item \textbf{Cible marge nette / étudiant :} $>$ \textbf{35\%} après coûts mentors + infrastructure (objectif).
    \item \textbf{Diversification revenus :} formation (socle) + ISA (alignement) + B2B (upside).
\end{itemize}

\vspace{0.6em}
\textbf{Go-to-market.} « Building in Public » : repos GitHub, replays de code reviews, démonstrations de capstones, publications d’audits et d’outils. Conversion via \textbf{webinars}, \textbf{simulateur ROI}, \textbf{Discord}, puis sélection technique.

\vspace{0.6em}
\textbf{Demande partenaire / investisseur.} Soutien à l’industrialisation (contenus gold, cockpits, mentors), financement du fonds de garantie ISA (si activé), et partenariats B2B pour upskilling. Objectif : installer RBK 2.0 comme \textbf{hub tunisien} de talents exportables (software export) avec gouvernance et conformité renforcées.
