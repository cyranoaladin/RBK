\chapter{REFERENCES \& BIBLIOGRAPHIE}
\label{chap:references}

\section{Documentation Technique}
\begin{itemize}
    \item \textbf{Solana Docs} : \url{https://docs.solana.com}
    \item \textbf{Anchor Framework} : \url{https://www.anchor-lang.com}
    \item \textbf{OtterSec Blog} : \url{https://osec.io/blog}
\end{itemize}

\section{Rapports de l'Industrie}
\begin{itemize}
    \item \textbf{Electric Capital Developer Report (2023)}
    \item \textbf{Messari State of Solana (Q4 2024)}
\end{itemize}
\begin{itemize}
    \item \textbf{Solana Whitepaper (2017)} : Architecture Proof of History. \textit{Anatoly Yakovenko.}
    \item \textbf{Anchor Framework Docs} : Spécifications techniques du framework standard Solana.
    \item \textbf{Rust Book (The)} : Bible officielle du langage Rust. \textit{Steve Klabnik \& Carol Nichols.}
\section{Rapports de Marché (Secondaire)}
\begin{itemize}
    \item \textbf{Electric Capital Developer Report (2024)} : Croissance des écosystèmes développeurs (+400\% sur Solana).
    \item \textbf{HackerOne Security Report} : Salaires moyens des auditeurs Web3.
    \item \textbf{Superteam Earn Metrics} : Données sur les gains moyens en bounties (2023-2024).
\end{itemize}

\section{Outils Cités}
\begin{itemize}
    \item \textbf{Helius} : Observabilité Solana.
    \item \textbf{Trident} : Solana Fuzzing Framework.
    \item \textbf{Metaplex} : Standard NFT sur Solana.
\end{itemize}
