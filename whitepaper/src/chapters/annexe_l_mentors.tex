\chapter{STRATÉGIE MENTORAT \& TRAIN-THE-TRAINER}

La qualité de RBK 2.0 repose sur la qualité de son encadrement humain. Nous ne recrutons pas des "profs", mais des "Tech Leads" capables de guider des juniors.

\section{Le Pipeline "Train the Trainer"}

Pour assurer la scalabilité sans perte de qualité, RBK forme ses propres mentors parmi les meilleurs Alumni.

\begin{enumerate}
    \item \textbf{Sourcing :} Top 10\% des diplômés (Score Tech > 90/100 + Soft Skills A).
    \item \textbf{Shadowing (1 Cohorte) :} L'aspirant-mentor suit un mentor Senior pendant 3 mois. Il corrige les exercices simples et anime les Daily Stand-ups.
    \item \textbf{Certification Pédagogique :} Formation interne de 2 semaines sur :
    \begin{itemize}
        \item La méthode Socratique (répondre par une question).
        \item La gestion de crise émotionnelle (Protocole Anti-Burnout).
        \item La détection de triche par IA.
    \end{itemize}
    \item \textbf{Titularisation :} Prise en charge d'une Squad de 15 étudiants.
\end{enumerate}

\section{Modèle de Rémunération Incitatif}

Nous alignons les intérêts des mentors sur la réussite des étudiants.

\begin{table}[h]
    \caption{Grille de Rémunération Mentor (Junior $\to$ Lead)}
    \centering
    \small
    \begin{tabular}{|l|l|l|}
        \hline
        \textbf{Niveau} & \textbf{Fixe (Mensuel)} & \textbf{Variable (Performance)} \\ \hline
        \textbf{Junior Mentor} & 2 500 TND & 100 TND par étudiant validant le N1. \\ \hline
        \textbf{Senior Mentor} & 4 500 TND & 2\% du Pool ISA de sa cohorte (si placement > 90\%). \\ \hline
        \textbf{Lead Instructor} & 7 000 TND & Part de l'EBITDA annuel (BSPCE/Tokens). \\ \hline
    \end{tabular}
\end{table}

\section{Plan de Relève et Continuité}

Pour éviter le "Bus Factor" (départ d'un instructeur clé) :
\begin{itemize}
    \item \textbf{Binômes Rotatifs :} Chaque module critique (ex: Rust Advanced) est maîtrisé par au moins 2 mentors Seniors.
    \item \textbf{Documentation "Playbook" :} Chaque cours dispose d'un guide "Teacher's Notes" détaillant les points de friction habituels et les métaphores clés.
    \item \textbf{Guest Lecturers :} Bassin de 5 experts externes (CTO partenaires) activables pour des masterclasses ponctuelles ou des remplacements d'urgence.
\end{itemize}
